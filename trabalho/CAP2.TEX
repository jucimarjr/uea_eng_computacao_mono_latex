<<<<<<< HEAD:trabalho/CAP2.TEX
\chapter{Manual do Trabalho de Conclusão de Curso - TCC - I e II}
=======
\chapter{Manual do Trabalho de Conclus\~ao de Curso - TCC - I e II}
>>>>>>> develop:trabalho/CAP2.TEX
\label{cap:manual TCC}


\section{Trabalho de Conclusão de Curso (TCC)}

<<<<<<< HEAD:trabalho/CAP2.TEX
O TCC é o último e mais importante trabalho de disciplina a ser desenvolvido pelo aluno, individualmente, 
no Curso de Graduação de Engenharia da Computação. O aluno, orientado por um professor, terá oportunidade de demonstrar e por em prática 
os conhecimentos adquiridos durante o curso, além de aperfeiçoar e comprovar o aprendizado teórico e metodologias que lhe foram ensinadas.
=======
O TCC \'e o último e mais importante trabalho de disciplina a ser desenvolvido pelo aluno, individualmente, 
no Curso de Graduação de Engenharia da Computação. O aluno, orientado por um professor, terá oportunidade de demonstrar e por em prática 
os conhecimentos adquiridos durante o curso, al\'em de aperfeiçoar e comprovar o aprendizado teórico e metodologias que lhe foram ensinadas.
>>>>>>> develop:trabalho/CAP2.TEX

O TCC consiste em duas etapas a serem realizadas em semestres distintos:

	\begin{enumerate}
		\item \textbf{TCC}-I: Elaboração e defesa da monografia.
		O aluno deve documentar, na forma de monografia, toda a fundamentação teórica e modelagem (se aplicável), de acordo com o tema proposto.
		
		\item \textbf{TCC}-II: Após aprovação do TCC-I o aluno deverá concluir a monografia ao implementar a modelagem proposta ou outro instrumento que possibilite a 			avaliação do trabalho face aos objetivos definidos. A defesa inclui a documentação finalizada e demonstração do que foi desenvolvido (implementação).
	\end{enumerate}
	
O TCC, do curso de Engenharia da Computação, corresponde a um total de quatro (04) cr\'editos, a serem efetivados mediante matrícula, sendo que dois (02) cr\'editos serão cumpridos na disciplina TCC-I e dois (02) cr\'editos na disciplina  TCC-II. 


\subsection{Do Professor da Disciplina}

<<<<<<< HEAD:trabalho/CAP2.TEX

\subsection{Do Professor da Disciplina}

=======
>>>>>>> develop:trabalho/CAP2.TEX
O TCC funciona como uma disciplina e, portanto, possui um professor responsável, o qual encaminha para avaliação os trabalhos finais (TCC-I e TCC-II) e controla as notas finais. 

Compete ao professor da disciplina:

	\begin{itemize}
		\item Estabelecer reunião inicial com os alunos e professores para expor as normas do TCC e dar-lhes ciência desse documento

		\item Divulgar e fazer cumprir as normas referentes ao TCC
		
		\item Divulgar o calendário referente às atividades do TCC

<<<<<<< HEAD:trabalho/CAP2.TEX
		\item Penalizar o aluno pelo não cumprimento dos prazos determinados no calendário de atividades do TCC; (0,2 pts, na média final, por dia de atraso na entrega 		da proposta e monografia)
=======
		\item Penalizar o aluno pelo não cumprimento dos prazos determinados no calendário de atividades do TCC; (0,2 pts, na m\'edia final, por dia de atraso na entrega 		da proposta e monografia)
>>>>>>> develop:trabalho/CAP2.TEX
		
		\item Divulgar aos alunos a relação de professores e suas respectivas linhas de pesquisa para exercerem a atividade de orientação
		
		\item Coordenar a formação das duplas orientador/orientando

		\intem Coordenar o processo de substituição orientador/orientando
		
		\item Coordenar o processo de constituição das bancas e definir o cronograma de apresentação dos trabalhos
		
		\item Encaminhar aos membros da banca o respectivo trabalho para avaliação
		
		\item Convidar, quando possível, um avaliador externo para integrar a banca
		
		\item Providenciar os documentos necessários ao processo de avaliação

		\item Providenciar as declarações aos professores participantes da banca, bem como do orientador
		
<<<<<<< HEAD:trabalho/CAP2.TEX
		\item Encaminhar uma cópia definitiva do trabalho (TCC-II) à Biblioteca
=======
		\item Encaminhar uma c\´opia definitiva do trabalho (TCC-II) \´a Biblioteca
>>>>>>> develop:trabalho/CAP2.TEX
		
	\end{itemize}


\subsection{Professor Orientador}

O professor orientador tem a função de ajudar o aluno no direcionamento do seu trabalho, sem, entretanto, desenvolver partes desse 
trabalho para o aluno. O orientador, apenas sugere caminhos que o aluno deverá seguir, acompanha seu trabalho, motivando e corrigindo eventuais erros.

Antes de apresentar o TCC-I ou TCC-II, o aluno deve submetê-lo previamente, \textbf{e obrigatoriamente}, à apreciação de seu orientador. 
Dado o aval do mesmo, a proposta poderá ser encaminhada e apresentada para avaliação.

Compete ao Professor Orientador:

	\begin{itemize}
		\item Informar ao professor da disciplina a linha de pesquisa que irá atuar
	
		\item Orientar a elaboração do Trabalho de Conclusão
	
<<<<<<< HEAD:trabalho/CAP2.TEX
		\item Auxiliar o aluno na resolução de problemas conceituais, técnicos e de relacionamento decorrentes da atividade
	
		\item Estabelecer o plano e cronograma de trabalho em conjunto com o orientando
	
		\item Informar o orientando sobre as normas, procedimentos e critérios de avaliação respectivos
=======
		\item Auxiliar o aluno na resolução de problemas conceituais, t\'ecnicos e de relacionamento decorrentes da atividade
	
		\item Estabelecer o plano e cronograma de trabalho em conjunto com o orientando
	
		\item Informar o orientando sobre as normas, procedimentos e crit\'erios de avaliação respectivos
>>>>>>> develop:trabalho/CAP2.TEX
	
		\item Informar ao aluno, caso haja atraso no cronograma de trabalho ou o não cumprimento das orientações, se o trabalho tem condições ou não de ser 
		encaminhado para avaliação
	
		\item liberar o trabalho para que haja a apresentação do aluno bem como informar ao professor da disciplina quanto à apresentação do aluno (Anexo C)
	
		\item Rubricar as 3 (três) vias encaminhadas para avaliação (TCC-I ou TCC-II) quando estiver ciente e de acordo, 
		conforme suas orientações, do material entregue
	
		\item Presidir a banca examinadora do trabalho por ele orientado
	
		\item Comunicar ao professor da disciplina situações que exijam providências, assim que ocorrerem.
	\end{itemize}


\subsection{Co-orientador}

	\begin{itemize}	
		\item Será solicitado formalmente pelo orientador a Professor da disciplina e será designado por esse para atender questão específica do trabalho
		
		\item Trabalhará em conjunto ao orientador e desempenhará papel solicitado pelo mesmo.
	\end{itemize}


\subsection{Orientando}

Compete ao Orientando:

	\begin{itemize}
		\item Comparecer às reuniões marcadas pelo professor da disciplina sobre o Trabalho de Conclusão
		
		\item Escolher a temática a ser trabalhada em consonância com as linhas de pesquisa do curso
		
		\item Contatar professor para definir orientador e informar a Coordenação do Projeto (entrega do Anexo-B)
		
		\item Cumprir as datas limites determinadas no calendário de atividades do TCC. O não cumprimento dos prazos será penalizado com perda de pontuação; 
<<<<<<< HEAD:trabalho/CAP2.TEX
		(0,2 pts, na média final, por dia de atraso na entrega da monografia)
=======
		(0,2 pts, na m\'edia final, por dia de atraso na entrega da monografia)
>>>>>>> develop:trabalho/CAP2.TEX
		
		\item Comparecer às orientações sobre o trabalho; o não comparecimento de três (03) orientações seguidas implica em reprovação por falta
		
		\item Seguir as orientações do professor designado à orientação
		
		\item Cumprir o plano e o cronograma de trabalho elaborado em conjunto com o professor-orientador
		
		\item Comunicar ao professor da disciplina toda e qualquer situação que possa comprometer, de alguma forma, o processo de elaboração, bem como, a conclusão do 			trabalho o quanto antes, para que a coordenação possa analisar o ocorrido e tomar as providências cabíveis
		
		\item Comparecer perante a banca na data, hora e local estabelecido para a realização da sessão de avaliação do TCC
	\end{itemize}


\subsection{Os Acompanhamentos de Orientação}

As reuniões de orientação deverão ser documentadas conforme modelo presente no Anexo A e serão entregues ao professor da disciplina no dia da entrega da carta (ANEXO A) solicitando defesa de TCC.
Tanto professor orientador como orientando deverão ter uma cópia dos acompanhamentos de orientação.
	

\subsection{A Banca Examinadora}

A banca examinadora do TCC-I e TCC –II, deverá ser composta por, no mínimo, 3 professores. A banca será constituída pelo 
professor orientador e por dois outros professores. Se houver co-orientação, o professor co-orientador pode compor a banca, contudo sua avaliação não computará nota para o alunosua avaliação não computará nota para o aluno 

<<<<<<< HEAD:trabalho/CAP2.TEX
Os membros da banca examinadora poderão sugerir alterações no trabalho (parte escrita e/ou implementação). Para o TCC-I as alterações deverão ser feitas, com o acompanhamento do orientador, para que sejam incluídas no trabalho e avaliadas no TCC-II. Para o TCC-II, estas deverão ser feitas até duas semanas depois da apresentação (ver data limite), supervisionadas pelo professor-orientador, para constar no(s) volume(s) final(is) do TCC, que ficará à disposição na biblioteca
=======
Os membros da banca examinadora poderão sugerir alterações no trabalho (parte escrita e/ou implementação). Para o TCC-I as alterações deverão ser feitas, com o acompanhamento do orientador, para que sejam incluídas no trabalho e avaliadas no TCC-II. Para o TCC-II, estas deverão ser feitas at\'e duas semanas depois da apresentação (ver data limite), supervisionadas pelo professor-orientador, para constar no(s) volume(s) final(is) do TCC, que ficará à disposição na biblioteca
>>>>>>> develop:trabalho/CAP2.TEX

O volume final para arquivamento (TCC-II) só será aceito pela coordenação de TCC se estiver validado pelo professor orientador, indicando sua concordância com o conteúdo do mesmo, e a assinatura do aluno


\subsection{Seminários}
Conforme calendário os seminários destinados aos alunos matriculados em TCC, abordam temas que auxiliarão na elaboração do documento escrito e na defesa.

Serão e seminários:
	\begin{itemize}
		\item Seminário I – Estrutura do Trabalho de Conclusão de Curso

		\item Seminário II – Normas ABNT

		\item Seminário III – Apresentação do TCC (defesa e material)
	\end{itemize}

<<<<<<< HEAD:trabalho/CAP2.TEX
A participação do aluno nos seminários é um dos critérios que consta na ata de avaliação. O não comparecimento acarretará perda de 0,2 pt por seminário.
=======
A participação do aluno nos seminários \'e um dos crit\'erios que consta na ata de avaliação. O não comparecimento acarretará perda de 0,2 pt por seminário.
>>>>>>> develop:trabalho/CAP2.TEX


\subsection{As Datas Limite}

As datas limites serão estabelecidas e divulgadas de acordo com o calendário acadêmico de cada período acadêmico.


\subsection{Nota Final}

Para aprovação do aluno no TCC, o mesmo deverá:

	\begin{itemize}
		\item Atender à exigência da frequência mínima de 75$'%'$ (setenta e cinco) às orientações. 
		A frequência do aluno será validada a partir do formulário de acompanhamento de reunião de orientação os quais devem ser preenchidos a cada acompanhamento, 			pelo orientador e pelo aluno

		\item Obter, no mínimo, grau 6,0 (seis). Este grau será composto pela m\'edia aritm\'etica das avaliações dos membros da banca examinadora. Cada membro da banca 			receberá uma planilha com itens a avaliar (por notas). Ao t\'ermino da defesa será preenchida uma ata final de avaliação de TCC constando a m\'edia final do aluno.
		Caso o aluno não alcance grau mínimo seis (6,0) deverá matricular-se novamente na disciplina para desenvolver novamente o trabalho ou concluir o 			desenvolvimento do mesmo
	\end{itemize}


\section{Trabalho de Conclusão de Curso I (TCC-I)}

O aluno, em parceria com um professor orientador, deve delimitar um tema, a ser abordado, dentro das linhas de pesquisa do curso.

Deve então dar início à documentação de seu trabalho elaborando uma monografia com os capítulos contendo a fundamentação teórica e modelagem da implementação do trabalho proposto. Ao final do semestre defendê-lo à uma banca examinadora.

\subsection{Estrutura do TCC-I}

Na monografia, o aluno deverá documentar seu trabalho para ser arquivado e, no futuro, referenciado por outras pessoas, lembrando sempre que Trabalho de Conclusão de Curso deve ser escrito tendo em vista uma metodologia científica. 

A monografia deve seguir a seguinte estrutura: 
	
	\indent \textbf{Capa} \\
	\indent \textbf{Folha de Rosto} \\
	\indent \textbf{Ficha para Catalogação} (deve ser impressa no verso da folha de rosto) \\
	\indent \textbf{Epígrafe} (opcional) \\
	\indent \textbf{Dedicatória} (opcional) \\
	\indent \textbf{Agradecimentos} (opcional) \\
	\indent \textbf{Resumo} \\
	\indent \textbf{Abstract} (resumo em inglês) \\
	\indent \textbf{Sumário} \\
	\indent \textbf{Lista de Figuras} (opcional) \\
	\indent \textbf{Lista de Tabelas} (opcional) \\
	\indent \textbf{Lista de Abreviaturas e siglas} \\
	\indent \textbf{Introdução} \\
	\indent \textbf{Desenvolvimento} \\
	\indent \textbf{Conclusão} \\
	\indent \textbf{Referências Bibliográficas} \\
	\indent \textbf{Obras Consultadas} \\
	\indent \textbf{Anexos e/ou Apêndices} (opcional) \\\\

\textbf{Introdução} - \'e o primeiro capítulo da monografia. Apresenta o contexto do trabalho proposto com a definição do problema, os objetivos (geral e específicos), os motivos que levaram à decisão de se abordar o tema e a organização do trabalho.

\textbf{Desenvolvimento} - corresponde aos demais capítulos da monografia, que descrevem sobre o tema proposto, revisão da literatura, metodologia aplicada, ferramentas e modelagem (se aplicável) do trabalho a ser implementado.

<<<<<<< HEAD:trabalho/CAP2.TEX
\textbf{Conclusão} - como o TCC-I é o início da monografia não será possível uma conclusão, portanto devem ser apresentadas as dificuldades encontradas, até o momento no trabalho, e resultados esperados do trabalho proposto.
=======
\textbf{Conclusão} - como o TCC-I \'e o início da monografia não será possível uma conclusão, portanto devem ser apresentadas as dificuldades encontradas, at\'e o momento no trabalho, e resultados esperados do trabalho proposto.
>>>>>>> develop:trabalho/CAP2.TEX

A formatação (margens, espaçamentos, citações, paginação, etc.) de todo o documento, deve estar voltada para um trabalho científico, portanto, 
os alunos devem seguir o modelo de monografia adotado pelo curso e disponível no site do mesmo. 
Solicitamos ainda aos alunos que utilizem as obras abaixo:

	\begin{itemize}	
<<<<<<< HEAD:trabalho/CAP2.TEX
		\item  FURASTÉ, Pedro Augusto. Normas Técnicas para o Trabalho Científico (Nova ABNT). 14ª edição. Porto Alegre, 2006.
=======
		\item  FURAST\'e, Pedro Augusto. Normas T\'ecnicas para o Trabalho Científico (Nova ABNT). 14ª edição. Porto Alegre, 2006.
>>>>>>> develop:trabalho/CAP2.TEX

		\item SILVA, Edna Lúcia da. Metodologia da Pesquisa e Elaboração de Dissertação – 3ª ed. rev. e atual. – Florianópolis: Laboratório de Ensino a Distância da 			UFSC, 2001.

		\item BRASIL, ABNT – Associação Brasileira de Normas T\'ecnicas. NBR 14724.

		\item BRASIL, ABNT – Associação Brasileira de Normas T\'ecnicas. NBR 10520.

<<<<<<< HEAD:trabalho/CAP2.TEX
		\item BRASIL, ABNT – Associação Brasileira de Normas Técnicas. NBR 6023.
=======
		\item BRASIL, ABNT – Associação Brasileira de Normas T\'ecnicas. NBR 6023.
>>>>>>> develop:trabalho/CAP2.TEX
	\end{itemize}
	

\subsection{Avalição do TCC-I}

<<<<<<< HEAD:trabalho/CAP2.TEX
O TCC-I deverá ser apresentado perante uma banca examinadora a ser definida pelo professor da disciplina, para a qual o aluno apresentará seu trabalho, desde a justificativa do problema que o levou a desenvolvê-lo até as discussões do material levantado.
=======
O TCC-I deverá ser apresentado perante uma banca examinadora a ser definida pelo professor da disciplina, para a qual o aluno apresentará seu trabalho, desde a justificativa do problema que o levou a desenvolvê-lo at\'e as discussões do material levantado.
>>>>>>> develop:trabalho/CAP2.TEX

O aluno terá 25 (vinte e cinco) minutos para defesa de sua proposta, onde utilizará os recursos audiovisuais que achar necessário e serão utilizados mais 10 (dez) minutos para responder aos questionamentos de cada membro da banca avaliadora. A banca será constituída pelo professor orientador (presidente) e por dois outros professores, podendo ser um convidado externo. Avaliado o trabalho escrito e “ouvidas” as sugestões da banca, o aluno deverá fazer as modificações necessárias.

No TCC-I caso o aluno não alcance grau mínimo 6,0 (seis) deverá matricular-se novamente na disciplina pra desenvolver novamente o trabalho (ou concluir o desenvolvimento do mesmo)  em TCC-I.
<<<<<<< HEAD:trabalho/CAP2.TEX
É necessário que o aluno seja aprovado em TCC-I para a conclusão do trabalho em TCC-II. 
=======
\'e necessário que o aluno seja aprovado em TCC-I para a conclusão do trabalho em TCC-II. 
>>>>>>> develop:trabalho/CAP2.TEX


\section{Trabalho de Conclusão de Curso II (TCC-II)}

<<<<<<< HEAD:trabalho/CAP2.TEX
O aluno deverá por em prática a modelagem apresentada em TCC-I além de concluir a monografia (implementação, resultados obtidos, etc.). Haverá nova defesa da documentação e demonstração do que foi desenvolvido (implementação) de acordo com esta documentação.
=======
O aluno deverá por em prática a modelagem apresentada em TCC-I al\'em de concluir a monografia (implementação, resultados obtidos, etc.). Haverá nova defesa da documentação e demonstração do que foi desenvolvido (implementação) de acordo com esta documentação.
>>>>>>> develop:trabalho/CAP2.TEX


\subsection{Estrutura do TCC-II}

Apresenta a mesma estrutura do TCC-I. Entretanto, no TCC-II, o aluno irá complementar a monografia de acordo com as solicitações feitas pela banca examinadora (na defesa do TCC-I) e com tópicos relacionados à sua implementação. 

A monografia deve seguir a seguinte estrutura: 

	\indent \textbf{Capa} \\
	\indent \textbf{Folha de Rosto} \\
	\indent \textbf{Ficha para Catalogação} (deve ser impressa no verso da folha de rosto) \\
	\indent \textbf{Epígrafe} (opcional) \\
	\indent \textbf{Dedicatória} (opcional) \\
	\indent \textbf{Agradecimentos} (opcional) \\
	\indent \textbf{Resumo} \\
	\indent \textbf{Abstract} (resumo em inglês) \\
	\indent \textbf{Sumário} \\
	\indent \textbf{Lista de Figuras} (opcional) \\
	\indent \textbf{Lista de Tabelas} (opcional) \\
	\indent \textbf{Lista de Abreviaturas e siglas} \\
	\indent \textbf{Introdução} \\
	\indent \textbf{Desenvolvimento} \\
	\indent \textbf{Conclusão} \\
	\indent \textbf{Referências Bibliográficas} \\
	\indent \textbf{Obras Consultadas} \\
	\indent \textbf{Anexos e/ou Apêndices} (opcional) \\\\


\textbf{Introdução} - \'e o primeiro capítulo da monografia. Apresenta o contexto do trabalho proposto com a definição do problema, os objetivos (geral e específicos), os motivos que levaram à decisão de se abordar o tema e a organização do trabalho.

\textbf{Desenvolvimento} - corresponde aos demais capítulos da monografia, que descrevem sobre o tema proposto, revisão da literatura, metodologia aplicada, ferramentas e modelagem (se aplicável) do trabalho a ser implementado.

\textbf{Conclusão} - se os objetivos foram atingidos, dificuldades encontradas e sugestões para trabalhos futuros.

A formatação (margens, espaçamentos, citações, paginação, etc.) de todo o documento, deve estar voltada para um trabalho científico, portanto, 
os alunos devem seguir o modelo de monografia adotado pelo curso e disponível no site do mesmo. 
Solicitamos ainda aos alunos que utilizem as obras abaixo:

	\begin{itemize}	
<<<<<<< HEAD:trabalho/CAP2.TEX
		\item  FURASTÉ, Pedro Augusto. Normas Técnicas para o Trabalho Científico (Nova ABNT). 14ª edição. Porto Alegre, 2006.
=======
		\item  FURAST\'e, Pedro Augusto. Normas T\'ecnicas para o Trabalho Científico (Nova ABNT). 14ª edição. Porto Alegre, 2006.
>>>>>>> develop:trabalho/CAP2.TEX

		\item SILVA, Edna Lúcia da. Metodologia da Pesquisa e Elaboração de Dissertação – 3ª ed. rev. e atual. – Florianópolis: Laboratório de Ensino a Distância da 			UFSC, 2001.

		\item BRASIL, ABNT – Associação Brasileira de Normas T\'ecnicas. NBR 14724.

		\item BRASIL, ABNT – Associação Brasileira de Normas T\'ecnicas. NBR 10520.

		\item BRASIL, ABNT – Associação Brasileira de Normas T\'ecnicas. NBR 6023.
	\end{itemize}


\subsection{Avaliação do TCC-II}

O TCC-II deverá ser apresentado perante uma banca examinadora a ser definida pelo professor da disciplina, para a qual o aluno apresentará seu trabalho, desde a justificativa do problema que o levou a desenvolvê-lo at\'e as discussões do material levantado e a conclusão.

Esta apresentação deverá ser, necessariamente, oral e descritiva, onde o aluno deverá tamb\'em na parte oral resumir as principais funções do sistema, o modo como será usado na organização ou ambiente e os seus benefícios.

Para esta apresentação oral, o aluno deverá preparar o que irá falar e utilizar recursos didáticos, considerando o tempo de 45 minutos; cada membro da banca avaliadora terá 10 minutos para questionamentos.

Não serão aceitas justificativas para a não demonstração das implementações, implicando assim em reprovação.

As apresentações dos TCCs são abertas ao público interessado. Sugere-se, fortemente, que os alunos de TCC-I assistam às bancas de seus colegas de TCC-II, como experiência.

Caso o aluno não alcance a nota mínima de 6,0 pontos em TCC-II, deverá matricular-se, novamente na disciplina, no próximo semestre.
	
A nota do aluno só será lançada mediante entrega de 2 cópias da versão revisada com visto do orientador, já incluindo as modificações sugeridas pela banca, no formato final (impresso em jato de tinta ou laser), em capa dura, na cor preta, bem como um CD com a versão final do trabalho e os produtos resultantes da pesquisa, quando for o caso.

	
