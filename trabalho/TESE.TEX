\documentclass{dcctese}
\usepackage[latin1,utf8]{inputenc}  

\usepackage[brazil]{babel}   
\usepackage[T1]{fontenc}   


%\usepackage[portuguese]{babel}
%\usepackage[utf8]{inputenc}
%\usepackage[T1]{fontenc}


%\usepackage{psfig}
\usepackage{a4wide}
\usepackage{comment}
\usepackage[pdftex]{graphicx,color}
\usepackage{graphicx}
\usepackage{graphics}
\usepackage{cite}
\usepackage{longtable}
\usepackage{float}
\usepackage{fancyvrb}
\usepackage{fancyhdr}
\usepackage{setspace}
\usepackage{amsmath}
\usepackage{lscape}

\graphicspath{{../img/}} %path da pasta que contem as imagens

\renewcommand{\topfraction}{1}
\renewcommand{\bottomfraction}{1}
\renewcommand{\floatpagefraction}{1}
\renewcommand{\textfraction}{0}
%\renewcommand{\baselinestretch}{2}  
\doublespacing %espa�amento duplo
\sloppy

\floatstyle{plain}  %%% tipos: plain, boxed, ruled
\newfloat{codigo}{tbp}{lop}[section]
\floatname{codigo}{C\'{o}digo}

%%% nome para ser usado no sum�rio

\newcommand{\listofcodename}{Lista de C\'{o}digos}

\begin{document}

%\onehalfspacing

% folha de titulo / capa------------------------------------------------------------------------------------------------------------------------------------

\thispagestyle{empty}



\title{ \textbf{DESENVOLVIMENTO DE PADR\~{A}O PARA MONOGRAFIAS DE ENGENHARIA DE COMPUTA\c{C}\~{A}O DA UEA}}

\author{ \bf UNIVERSIDADE DO ESTADO DO AMAZONAS - UEA\\[12pt] \bf ESCOLA SUPERIOR DE TECNOLOGIA \\[12pt] \bf ENGENHARIA DE COMPUTA\c{C}\~AO\\ [96pt] NOME DO ALUNO}

%\author{ \bf UNIVERSIDADE DO ESTADO DO AMAZONAS - UEA \\[12pt] \bf ESCOLA SUPERIOR DE TECNOLOGIA \\[12pt] \bf ENGENHARIA DE COMPUTA\c{C}\~{A}O \\
%	[96pt] \bf LANIER MENEZES DOS SANTOS}

\maketitle

\begin{center}
\large Manaus\\
\large 2010
\end{center}

\newpage

% folha de Rosto----------------------------------------------------------------------------------------------------------------------------------------------
\input{folha_de_rosto.tex}
\newpage

% ficha Catalogr�fica------------------------------------------------------------------------------------------------------------------------------------------

\pagenumbering{roman}
\setcounter{page}{2}
\textit{ \textbf{\\ Universidade do Estado do Amazonas - UEA\\Escola Superior de Tecnologia - EST} }
\textit{\\Reitor:\\ \textbf{Nome do Reitor Vigente}\\Vice-Reitor:\\ \textbf{Nome do Vice-Reitor}}
\\
\textit{
Diretor da Escola Superior de Tecnologia:\\ 
\textbf{Nome completo do atual diretor da unidade}}
\\
\textit{
Coordenador do Curso de Engenharia de Computa\c{c}\~{a}o:\\
\textbf{Nome do atual coordenador}}
\\
\textit{
Coordenador da Disciplina Trabalho de Conclus\~ao de Curso:\\
\textbf{Nome do professor da disciplina}}
\\[12pt]
\textit{
Banca Avaliadora composta por: \hfill Data da Defesa: \rule{.5cm}{.1mm}/\rule{.5cm}{.1mm}/\rule{.9cm}{.1mm}.\\
}
\textit{ 
\textbf{Prof. T\'{i}tulo. Nome completo do professor} (Orientador)\\
\textbf{Prof. T\'{i}tulo. Nome completo do professor} \\
\textbf{Prof. T\'{i}tulo. Nome completo do professor} \\
}

\begin{center}\large \bf CIP - Cataloga\c{c}\~{a}o na Publica\c{c}\~{a}o\end{center}
\begin{center}
	\fbox{
		\parbox{17cm}{
			\begin{minipage}{16cm} 
				<C\'{o}digo CIP> \hspace*{1cm} <\'{U}ltimo Nome>, <Primeiro Nome>\\[12pt]
				\hspace*{2cm} \parbox{14cm}{
				\hspace*{0.5cm}<T\'{i}tulo da Monografia>/ 
				<Primeiro> <\'{U}ltimo Nome>; [orientado por] Prof. <T\'{i}tulo>. <Nome do Professor> - Manaus: UEA, <Ano>.\\
				\hspace*{0.5cm}240 p.: il.; 30cm\\
				\hspace*{0.5cm}Inclui Bibliografia\\
				\hspace*{0.5cm}Trabalho de Conclus\~{a}o de Curso (Gradua\c{c}\~{a}o em Engenharia de Computa\c{c}\~{a}o).
				Universidade do Estado do Amazonas, <Ano>.\\[12pt]
				\hspace*{8cm} CDU: \hrulefill}
			\end{minipage}
		}
	}
\end{center}
\newpage

% folha de aprova��o----------------------------------------------------------------------------------------------------------------------------------
\begin{center}
\bf LANIER MENEZES DOS SANTOS\\[1.5 cm]
\end{center}

\begin{center}
\bf DESENVOLVIMENTO DE PADR\~{A}O PARA MONOGRAFIAS DE ENGENHARIA DE COMPUTA\c{C}\~{A}O DA UEA\\[1.5cm]
\end{center}

\hspace*{8cm}
\begin{minipage}{8cm} 

Trabalho de Conclus\~{a}o de Curso apresentado \`{a} 
banca avaliadora do Curso de Engenharia de Computa\c{c}\~{a}o, 
da Escola Superior de Tecnologia, da Universidade do Estado do Amazonas, 
como pr\'e-requisito para obten\c{c}\~{a}o do t\'{\i}tulo de Bacharel em
Engenharia de Computa\c{c}\~{a}o.\\

\large \bf Aprovado em: \rule{.5cm}{.1mm}/\rule{.5cm}{.1mm}/\rule{.9cm}{.1mm}.
\end{minipage} 

BANCA EXAMINADORA\\[12 pt]

\noindent \hrulefill \hspace*{6cm} \\
\noindent \textbf{Prof. Jucimar Maia da Silva J\'{u}nior, Mestre}\\
\textit{UNIVERSIDADE DO ESTADO DO AMAZONAS}\\[0.5cm]

\noindent \hrulefill \hspace*{6cm} \\
\noindent \textbf{Prof. Nome do Professor, T\'{\i}tulo do Prof.}\\
\textit{UNIVERSIDADE DO ESTADO DO AMAZONAS}\\[0.5cm]

\noindent \hrulefill \hspace*{6cm} \\
\noindent \textbf{Prof. Nome do Professor, T\'{\i}tulo do Prof.}\\
\textit{UNIVERSIDADE DO ESTADO DO AMAZONAS}\\
\let\cleardoublepage\clearpage
%\newpage
% agradecimentos------------------------------------------------------------------------------------------------------------------------------------------------------
% use o prefacio para agradecimentos, dedicat�rias, ...
% o arquivo prefacio deve come�ar com \chapter*{Pref\'{a}cio}
\hspace*{8cm}
\begin{minipage}{8cm}

\vfill
\vskip 20em
\chapter*{Agradecimentos}  %{Prefácio}
Ao finalizar este trabalho, ap\'{o}s dois anos e seis meses, tive o
prazer de contar com a amizade e o incentivo de pessoas que
tornaram mais suave este caminho. A elas, agrade\c{c}o:

  Aos meus pais e meu irm\~{a}o, que ficaram privados de minha
  companhia diversas vezes, mas sempre me incentivaram e
  apoioaram.

  \`{A} minha ``fam\'{\i}lia'', com quem pude dividir alegrias
  e ang\'{u}stias durante o tempo em que ficamos em na faculade.
\end {minipage}
\newpage

% resumo----------------------------------------------------------------------------------------------------------------------------------------------------------------------

\begin{center} \LARGE \bf Resumo \end{center} 
\vskip 4em

Phasellus fringilla nulla eget nunc adipiscing in volutpat enim bibendum. 
Aliquam et ante at ipsum molestie sodales. Pellentesque mattis venenatis metus, 
at tristique diam ullamcorper a. Nulla non risus et libero accumsan facilisis 
id ac justo. Ut eleifend placerat velit quis vehicula. 

Neste contexto, a metodologia proposta, que envolve a verificação formal
do mecanismo de reconhecimento de ondas eletrocardiogr\'aficas e
seqüências de ondas com a utilização da ferramenta de verificação
de modelos {\em Verus}, tem apresentado resultados concretos na
correção e projeto deste mecanismo de reconhecimento atrav\'es da
identificação e correção de falhas, tornando o sistema mais seguro e
confi\'avel.

Aenean metus lectus, iaculis id tincidunt quis, 
tincidunt ut dolor. Integer porttitor tincidunt augue sed condimentum. Mauris pellentesque 
vestibulum justo, vel pellentesque tellus suscipit in. Nulla eget sem augue. 
Vestibulum in sapien nec nibh accumsan ultricies. Aliquam varius consectetur lorem sed luctus. 
Nullam id ipsum ut ante tristique tempor.

A aplica\c{c}\~ao da metodologia permitiu que outras ondas fossem
reconhecidas pelo mecanismo de leitura de dados
eletrocardiogr\'aficos, de forma que fosse feito o
re-projeto do autômato para reconhecimento de ondas, bem como
fosse projetado um aut\^omato para o reconhecimento de sequ\^encias
de ondas, contribuindo de maneira concreta para o desenvolvimento
da aplicação biom\'edica.


Palavras Chave: Lorem, Ipsum

\newpage

% abstract---------------------------------------------------------------------------------------------------------------------------------------------------------------------

\begin{center} \LARGE \bf Abstract \end{center}
\vskip 4em
\input{ABSTRACT.TEX}

\newpage

% sum�rio-----------------------------------------------------------------------------------------------------------------------------------------------------------------------
\renewcommand{\contentsname}{Sum\'{a}rio}
\tableofcontents
\listoftables
\newpage

\listoffigures
\addcontentsline{toc}{chapter}{\listofcodename}
\listof{codigo}{\listofcodename}  % Lista de C�digos

\clearpage

% inicio da defini��o dos cabe�alhos--------------------------------------------------------------------------------------------------------------------------------------------
\pagestyle{fancy}
\renewcommand{\chaptermark}[1]{\markboth{#1}{}}
\renewcommand{\sectionmark}[1]{\markright{#1}}
\renewcommand{\headrulewidth}{0.5pt}
\newcommand{\rom}{\fontfamily{cmr}\fontseries{m}\fontsize{10}{12}\selectfont}
\fancyhf{} \fancyhead[LE,RO]{\rom\thepage}
\fancyhead[LO]{\rom\rightmark} \fancyhead[RE]{\rom\leftmark}
\fancypagestyle{plain}{
    \fancyhead{} % get rid of headers
    \renewcommand{\headrulewidth}{0pt} % and the line
 }
% fim dos cabe�alhos------------------------------------------------------------------------------------------------------------------------------------------------------------

% inicio do texto---------------------------------------------------------------------------------------------------------------------------------------------------------------

\pagenumbering{arabic}

% os arquivos de cada capitulo devem iniciar com \chapter{Nome do Capitulo}
% (o nome deve terminar com .tex)

%% Introdu��o
\chapter {Como Escrever Uma monografia}\label{cap:comoEscrever}

\section{Ao Candidato}

O texto abaixo foi livremente adaptado de \cite{COMER2010} para ajudar os alunos a escreverem suas monografias. 

Se você está se preparando para escrever uma monografia em uma área experimental da Engenharia da Computação. 
A menos que você tenha escrito muitos documentos formais antes, você tem uma surpresa: isso é difícil!

Existem dois caminhos possíveis para o sucesso:

\begin{itemize}
	\item Planejamento
	
	Poucas pessoas pegam esse caminho. As poucas que pegam, deixam a universidade tão rápido, 
	que eles malmente são notados. Se você quer fazer a impressão final e ter uma longa carreira como um 
	estudante de graduação, não escolha este caminho.
	
	\item Perseverança
	
	Tudo que você tem que fazer é sobreviver à sua banca julgadora. 
	A boa notícia é que eles são bem mais velhos que você, assim você pode adivinhar quem vai eventualmente 
	``expirar''  primeiro (morrer).
	A má notícia é que é que eles são mais experientes nesse jogo (afinal, eles perseveraram na frente da banca deles, não!?).
	
\end{itemize}
	
Aqui estão algumas linhas-guia que podem ajudá-lo quando você finalmente levar a sério escrever. 
A lista segue infinitamente; você provavelmente não vai querer ler isso tudo de uma vez. Mas, por favor, leia isso antes de 
escrever qualquer coisa.

\hfill

\section{A Idéia Geral}	

\begin{enumerate}
	
	\item Uma monografia é um documento formal onde o aluno descreve a realização de um trabalho técnico onde usou as técnicas e conceitos aprendidos durante o curso de 	graduação.

	\item Em geral, toda afirmação em uma monografia deve ser embasada ou por uma referência em literatura científica publicada ou por um trabalho original.
	Acima de tudo, uma monografia não repete os detalhes do pensamento crítico e análises encontradas nas fontes publicadas; usa o resultado como fato e 
	referencia o leitor às fontes para mais detalhes. 

	\item Cada sentença em uma monografia deve ser completa e correta gramaticalmente. Além do mais, a monografia deve satisfazer estritamente as 
	regras da gramática formal (e.x., sem contrações, sem coloquialismo, sem pronúncias erradas, sem jargão técnico indefinido, sem piadas escondidas e sem gírias,
	mesmo quando tais termos ou frases são de comum uso na língua falada). Realmente, a escrita de uma monografia deve ser um cristal limpo.
	Sombras de significados importam; a terminologia e a prosa devem fazer uma fina distinção. As palavras devem carregar exatamente o sentido pretendido, nada mais e nada 	menos.

	\item Cada afirmação em uma monografia deve ser correta e defensível no sentido lógico e científico. Acima de tudo, as discussões em uma monografia devem satisfazer
	a maioria das estritas regras de lógica aplicada à matemática e Engenharia.                       

\end{enumerate}

\section{O que se Deve Aprender do Exercício}

\begin{enumerate}

	\item Todo engenheiro precisa comunicar descobertas; a monografia fornece um treinamento para comunicação com outros engenheiros.

	\item Escrever uma monografia requer que o estudante pense profundamente, para organizar a discussão técnica, para reunir argumentos que convencerão outros engenheiro,
	e seguir as regras para uma rigorosa apresentação dos argumentos e discussões.

\end{enumerate}

\section{Regra do Polegar}

Boa escrita é essencial para uma monografia. Entretanto, boa escrita não pode compensar uma escassez de idéias ou conceitos.
Pelo contrário, uma apresentação limpa sempre expõe fraquezas.

\section{Definições e Terminologia}

\begin{enumerate}

	\item Cada termo técnico usado em uma monografia deve ser definido ou por uma referência à uma definição publicada anteriormente (para termos padrões com seus significados usuais)
	ou por uma precisa, não-ambígua definição que aparece antes do termo ser usado (para termos novos ou um termo padrão usado de maneira não usual).

	\item Cada termo deve ser usado de uma e única maneira por toda monografia.			

	\item A forma mais fácil de evitar uma longa série de definições é incluir uma afirmação: ``a terminologia usada no decorrer deste documento segue a mesma dada em [CITAÇÃO].''
	Então, só defina exceções.

	\item O capítulo introdutório pode dar o intuito (i.e., definições informais) dos termos fornecidos, os quais serão mais precisamente definidos depois.

\end{enumerate}

\section{Termos e Frase a Evitar}
	
\begin{itemize}

	\item Advérbios

		\indent Na maioria das vezes, são ``muito frequentemente usados demais''. Ao invés deles use palavras mais fortes. Alguém pode dizer, por exemplo, `` Escritores abusam de advérbios.''
	
	\item Piadas ou Trocadilhos

		\indent Esses não têm lugar em um documento formal.
	
	\item ``Ruim'', ``Bom'', ``Terrível'', ``Estúpido''

		\indent Uma monografia não faz julgamento moral. Use ``incorreto/correto'' para se referir à erros ou corretudes de fato.
		Use palavras precisas ou frases para avaliar qualidade (e.x. ``método A requer menos recurso computacional que método B''). Em geral, deve-se evitar todos os julgamentos qualitativos.
		
	\item ``Verdade'', ``Puro''

		\indent No mesmo senso de ``bom'' (é um julgamento).
		
	\item ``Perfeito''

		\indent Nada é.
		
	\item ``Uma solução ideal''

		\indent Você está julgando de novo.
		
	\item ``Hoje'', ``Tempos modernos''

		\indent Hoje é o ontem de amanhã.
		
	\item ``Logo''

		\indent Logo quanto? Hoje à noite? Próxima década?
		
	\item ``Estávamos surpresos ao ver ...''

		\indent Mesmo se você estivesse, e daí?
		
	\item ``Parece'', ``Aparentemente''

		\indent Não importa como algo aparenta.
	
	\item ``Parece mostrar''

		\indent Tudo o que importa são os fatos.
		
	\item ``Em termos de''

		\indent Normalmente vago.
		
	\item ``Baseado em'', ``X-baseado'', ``Como base de''

		\indent Cuidado, pode ser vago.
		
	\item ``Diferente''

		\indent	Não significa ``vários''. Diferente do que?
		
	\item ``Na luz de''

		\indent	Coloquial. 
		
	\item ``Um monte de''

		\indent	Vago & Coloquial
		
	\item ``Tipo de''

		\indent	Vago & Coloquial
		
	\item ``Algo como''

		\indent	Vago & Coloquial
		
	\item ``Mais ou Menos'' 

		\indent	Vago & Coloquial
		
	\item ``Número de''

		\indent	Vago, você quer dizer, ``alguns'', ``muitos'' ou ``a maioria''? Uma afirmação quantitativa é preferível.
		
	\item ``Devido a''

		\indent	Coloquial
		
	\item ``Provavelmente''

		\indent	Apenas se você souber a probabilidade estatística (se você sabe, afirme quantitativamente).
		
	\item ``Obviamente'', ``Claramente''

		\indent Tenha cuidado: Óbvio/Claro para todos?
		
	\item ``Simples''

		\indent Pode ter uma conotação negativa, como em ``simplório''.
		
	\item ``Junto com''

		\indent Use somente ``com''.
		
	\item ``Na verdade'', ``Realmente''

		\indent Defina os termos claramente para eliminar a necessidade de esclarecimento.
		
	\item ``O fato de''

		\indent Faz uma meta-sentença; reformule a frase.
		
	\item ``Isso'', ``Aquilo''

		\indent Como em ``Estas causa envolvem.'' Razão: ``Isso'' pode referir ao sujeito da sentença anterior, à toda sentença anterior, todo o parágrafo anterior, toda a seção anterior, etc. 
		Mais importante, pode ser interpretado no sentido correto ou no meta-sentido. Por exemplo: \textit{``X faz Y. Isso significa...''} o leitor pode assumir ``isso'' referindo ao 			
		\textit	{Y} ou ao fato de \textit{X faz Y}. Mesmo quando restrito (e.x., ``esta computação''), a frase é fraca e frequentemente ambígua.
		
	\item ``Você irá ler sobre isso''

		\indent A segunda pessoa não tem lugar em uma monografia.
		
	\item ``Eu vou descrever''

		\indent A primeira pessoa não tem lugar em uma monografia formal. Se auto-referência é essencial, escreva como ``Seção 10 descreve...''
		
	\item ``Nós'' como em ``Vemos que''

		\indent Uma armadilha a evitar. Razão: Quase toda sentença pode ser escrita para começar com ``nós'' porque `nós'' pode se referir: ao leitor e autor, ao autor e consultor, ao autor e grupo de pesquisa, engenheiros de computação, a toda comunidade de Engenharia da computação, ou algum outro grupo não especificado. 

	\item ``Esperançosamente, o programa''

		\indent Programas não tem esperança, não até serem implementados com sistemas de IA. Aliás, se você estiver escrevendo uma tese de IA, fale com outra pessoa: pessoas de IA tem seus próprios sistemas de regras.
		
	\item ``...um famoso pesquisador...''

		\indent Não importa quem disse ou fez. De fato, tais afirmações prejudicam o leitor.
		
	\item Tenha cuidado usando: ``poucos, maioria, todos, algum, cada'' 

		\indent Uma monografia é precisa. Se a sentença diz ``Maioria dos sistemas computacionais contém X'', você deve ser capaz de defender isso. Você tem certeza que conhece os fatos? Quantos computadores foram construídos e vendidos ontem?
		
	\item ``Deve'', ``Sempre''

		\indent Absolutamente?
		
	\item ``Deveria''

		\indent Quem disse isso?
		
	\item ``Prova'', ``Comprova''

		\indent Um matemático aceitaria que isso é uma prova?

	\item ``Pode'', ``Poderia''

		\indent Sua mãe provavelmente lhe disse a diferença.

\end{itemize}

\section{Voz}

Use construções ativas. Por exemplo, diga ``o sistema operacional inicia o dispositivo'' ao invés de ``o dispositivo é iniciado pelo sistema operacional.''

\section{Tempo Verbal}

Escreva no presente. ``O sistema escreve a página no disco e então usa o frame...'' ao invés de ``O sistema usará o frame depois de ter escrito a página no disco...''

\section{Defina Negações com Antecedência}

Exemplo: diga ``Nenhum bloco de dados espera na fila de saída'' ao invés de ``Um bloco de dados esperando saída não está na fila.''

\section{Gramática e Lógica}

Tenha cuidado pois o sujeito de cada sentença realmente faz o que o verbo diz q ele faz. 
Dizer ``Programas devem fazer chamada de processo usando a instrução X'' não é o mesmo que dizer ``Programas devem usar a instrução X quando chamam um procedimento.'' 
De fato, a primeira é evidentemente falsa! 
Outro exemplo: ``RPC requer programas para transmitir pacotes grandes'' não é o mesmo que ``RPC requer um mecanismo que permita programas transmitirem pacotes grandes.''

\section{Foco nos Resultados e não nas Pessoas/Circunstâncias em que Foram Obtidos}
		
``Depois de trabalhar oito horas no laboratório naquela noite. nós percebemos...'' não tem lugar na monografia. 
Não importa quando você percebeu isso, ou quanto tempo você trabalhou para obter a resposta. 
Outro exemplo: ``Jim e eu chegamos aos números mostrados na tabela 3 medindo...'' Ponha um agradecimento para Jim na monografia, mas não inclua nomes (nem mesmo o seu) no corpo principal. 
Você pode estar tentado a documentar uma longa série de experimentos que não produziram nada ou uma coincidência que resultou em sucesso. Evite completamente isso. 
Em particular, não documente aparentemente influências místicas (e.x., ``se aquele gato não tivesse rastejado pelo buraco no chão, poderíamos 
não ter descoberto o indicador de erro do fornecimento de energia na ponte de rede''). Nunca atribua tais eventos à causas místicas ou dê a 
entender que forças estranhas podem ter afetado seu resultado. 
Resumo: Prenda-se nos fatos evidentes. Descreva os resultados sem mencionar suas reações ou eventos que o ajudaram a alcançá-los.

\section{Evite Auto-Avaliação (Elogio e Crítica)}

Ambos os exemplos a seguir estão incorretos: ``O método esboçado na Seção 2 representa o maior avanço em design de sistemas distribuídos porque...'' 
``Embora a técnica na próxima seção não seja extraordinário,...''

\section{Referências à Trabalhos}

Sempre cita-se o artigo, não o autor. Assim, usa-se um verbo no singular para referir ao artigo, mesmo que tenha muitos autores. Por exemplo ``Johnson e Smith [Johnson and Smith1995] relata que...''
Evite a frase `` os autores afirmam que X''. O uso de ``afirmam'' lança dúvida em ``X'' porque referencia os pensamentos do autor ao invés dos fatos. Se você concorda ``X'' está correto, simplesmente escreva ``X'' seguido da referência. Se absolutamente deve referenciar um artigo ao invés do resultado, diga ``o artigo afirma que'' ou ``Johnson e Smith [Johnson and Smith1995] apresentam evidências que...''

\section{Conceito Vs. Exemplo}

Um leitor pode ficar confuso quando um conceito e um exemplo deste estão embaçados. 
Exemplos comuns incluem: um algoritmo e um programa particular que o implementa, uma linguagem de programação e um compilador, uma abstração geral e sua implementação 
particular em um sistema de computador, uma estrutura de dados e uma instância particular em memória.

\section{Terminologia para Conceitos e Abstrações}

Quando definir a terminologia para um conceito, tenha cuidado para decidir precisamente como a idéia se traduz para uma implementação. Considere a seguinte discussão:
\textit{Sistemas VM incluem um conceito conhecido como endereço de espaço. O sistema cria dinamicamente um endereço de espaço quando um programa precisa de um, e destrói um endereço de espaço 
quando o programa que criou o espaço terminar de usá-lo. Um sistema VM usa um pequeno, finito número para identificar cada endereço de espaço. 
Conceitualmente, entende-se que cada endereço de espaço deveria ter um novo identificador. Entretanto, se um sistema VM executa por um tempo que esgote todos os possíveis identificadores de endereços de espaço, 
ele deve reusar um número}
O ponto importante é que a discussão só faz sentido porque define ``endereço de espaço'' independente de ``identificador de endereço de espaço''. 
Se espera-se discutir as diferenças entre um conceito e sua implementação, as definições devem permitir tal distinção.

\section{Conhecimento Vs. Dados}

O fato que resulta de um experimento é chamado ``dado''. 
O termo ``conhecimento'' implica que o fato tenha sido analisado, condensado ou combinado com fatos de outros experimentos para produzir informação útil.

\section{Causa e Efeito}

Uma monografia deve separar cuidadosamente causa-efeito de simples correlações estatísticas. Por exemplo, mesmo se todos os programas de computador escritos no laboratório do Professor X requerem mais memória que os programas escritos no laboratório do Professor Y, isso pode não ter nada a ver com os professores ou laboratórios ou programadores (e.x., talvez as pessoas que trabalham no laboratório do Professor X estejam trabalhando em aplicações que requerem mais memória do que no laboratório do Professor Y).

\section{Descreva Somente Conclusões Comprovadas}   		

Deve-se ter cuidado para apenas escrever conclusões que as evidências suportam. Por exemplo, se programas executam muito mais lento no computador A do que no computador B, 
não pode-se concluir que o processador de A é mais lento que o de B a menos que se tenha anotado todas as diferenças entre os sistemas operacionais dos computadores, dispositivos de entrada e saída, 
tamanho de memória, memória cache, ou largura de banda do barramento interno. 
De fato, deve-se ainda abster-se de julgamentos a menos que se tenha os resultados de um experimento controlado (e.x., executando uma lista de vários programas muitas vezes, cada um quando o computador 
estiver ocioso). Mesmo se a causa de algum fenômeno parece óbvia, não pode-se dar uma conclusão sem sólida evidência embasada.

\section{Comércio e Ciência}

Em uma monografia, nunca se escreve conclusões sobre viabilidade econômica ou sucesso comercial de uma idéia/método, nem faz-se especulações sobre a história do desenvolvimento ou origens de
uma idéia. Um engenheiro deve permanecer objetivo sobre os méritos de uma idéia, independente de sua popularidade comercial. Em particular, um engenheiro nunca assume que o sucesso comercial é uma medida válida 
de mérito (muitos produtos populares não são nem bem projetados nem bem construídos). Assim, afirmações tais como ``mais de quatrocentos vendedores fazem produtos usando a técnica Y'' são irrelevantes em uma monografia.

\section{Política e Ciência}

Um engenheiro evita toda influência política quando está avaliando Idéias. Obviamente, não deveria importar se grupos governamentais, grupos políticos, grupos religiosos ou outras organizações aprovam uma idéia.
Mais importante e frequentemente despercebida, não importa se uma idéia uma idéia foi originada por um engenheiro que já tenha ganho um premio Nobel ou um aluno no primeiro ano de graduação. Deve-se avaliar a idéia independente da fonte.

\section{Organizações Canônicas}
 	
Em geral toda monografia deve definir o problema que motivou a pesquisa, contar por que este problema é importante, contar o que outros fizeram, descrever as novas contribuições, documentar os experimentos que validam a contribuição e fazer conclusões.
Não existe organização canônica para uma monografia; cada uma é única. Entretanto, novatos que escrevem uma monografia em uma área experimental da Engenharia da Computação podem achar os seguintes exemplos um bom ponto de início: 

\begin{itemize}
	
	\item \textbf{Capítulo 1: Introdução}

	Uma visão do problema; por que isso é importante; um resumo de um trabalho já existente e uma afirmação de suas hipóteses ou questões específicas a serem exploradas. Faça com que seja legível pra qualquer um.

	\item \textbf{Capítulo 2: Definições}

	Somente termos novos. Faça as definições precisas, concisas e não-ambíguas.

	\item \textbf{Capítulo 3: Modelo Conceitual}

	Descreva o conceito central que influencia o seu trabalho. Faça disso um ``tema'' que amarram todos os seus argumentos. Isso deveria fornecer uma resposta para a questão apresentada na introdução em um nível conceitual. Se necessário, adicione outro capítulo para dar um raciocínio adicional sobre o problema ou sua solução.

	\item \textbf{Capítulo 4: Medidas Experimentais}

	Descreva o resultado experimental que forneça evidências para embasar sua tese. Normalmente experimentos enfatizam prova-ou-conceito (demonstrando a viabilidade de um método/técnica) ou eficiência (demonstrando que um método/técnica proporciona uma performance melhor do que as que já existem)
	
	\item \textbf{Capítulo 5: Resultados e Consequências}

	Descreva variações, extensões ou outras aplicações da idéia central.

	\item \textbf{Capítulo 6: Conclusões}

	Resumo do que foi aprendido e como isso pode ser aplicado. Mencione as possibilidades para pesquisas futuras.

	\item \textbf{Resumo/Abstract}

	Um pequeno (poucos parágrafos) resumo da monografia. Descreva o problema e a abordagem da pesquisa. Enfatize as contribuições originais.

\end{itemize}

\section{Ordem Sugerida para Escrever}

A maneira mais fácil de construir uma monografia é de dentro para fora. Comece escrevendo os capítulos que descrevem sua pesquisa (3, 4 e 5 nas linhas acima). Colete termos como eles surgem no texto e dê uma definição para cada um.
Defina cada termo técnico,  mesmo que você o use de maneira convencional.

Organize as definições em um capítulo separado. Faça as definições precisas e formais. Reveja depois os capítulos para verificar que cada uso de termo técnico adere à uma definição. Depois de ler os capítulos do meio para verificar terminologia, escreva a conclusão. escreva a Introdução logo depois da Conclusão. Finalmente, complete o resumo/abstract.

\section{A Chave do Sucesso}

Aliás, existe uma chave para o sucesso: prática. ninguém nunca aprendeu a escrever lendo composições como esta. Ao invés disso, você precisa praticar, praticar, praticar. Todo dia.

\section{Pensamentos de Despedida}

Nos despedimos de você com as seguintes Idéias para meditar. Se não significarem nada para você agora, visite-as novamente depois de ter escrito sua monografia.

	

	\indent\indent Depois de grande dor, chega um pensamento formal.

		\indent \indent \indent --Emily Dickinson

		
	\indent\indent Um homem pode escrever a qualquer hora, se ele se mantiver persistente para tal.

		\indent \indent \indent --Samuel Johnson

		
	\indent\indent Permaneça perfeito até o final da estrada.

		\indent \indent \indent --Harry Lauder

	 
	\indent\indent Uma típica tese de Ph.D. é nada mais que transferir ossos de um cemitério para outro.

		\indent \indent \indent --Frank J. Dobie

	
	
	
	
	
	
	
	








	
	
	
	
	
	
	
	
	
	
	
	
	
	


%% T�tulo do Capitulo 2
<<<<<<< HEAD:trabalho/CAP2.TEX
\chapter{Manual do Trabalho de Conclusão de Curso - TCC - I e II}
=======
\chapter{Manual do Trabalho de Conclus\~ao de Curso - TCC - I e II}
>>>>>>> develop:trabalho/CAP2.TEX
\label{cap:manual TCC}


\section{Trabalho de Conclusão de Curso (TCC)}

<<<<<<< HEAD:trabalho/CAP2.TEX
O TCC é o último e mais importante trabalho de disciplina a ser desenvolvido pelo aluno, individualmente, 
no Curso de Graduação de Engenharia da Computação. O aluno, orientado por um professor, terá oportunidade de demonstrar e por em prática 
os conhecimentos adquiridos durante o curso, além de aperfeiçoar e comprovar o aprendizado teórico e metodologias que lhe foram ensinadas.
=======
O TCC \'e o último e mais importante trabalho de disciplina a ser desenvolvido pelo aluno, individualmente, 
no Curso de Graduação de Engenharia da Computação. O aluno, orientado por um professor, terá oportunidade de demonstrar e por em prática 
os conhecimentos adquiridos durante o curso, al\'em de aperfeiçoar e comprovar o aprendizado teórico e metodologias que lhe foram ensinadas.
>>>>>>> develop:trabalho/CAP2.TEX

O TCC consiste em duas etapas a serem realizadas em semestres distintos:

	\begin{enumerate}
		\item \textbf{TCC}-I: Elaboração e defesa da monografia.
		O aluno deve documentar, na forma de monografia, toda a fundamentação teórica e modelagem (se aplicável), de acordo com o tema proposto.
		
		\item \textbf{TCC}-II: Após aprovação do TCC-I o aluno deverá concluir a monografia ao implementar a modelagem proposta ou outro instrumento que possibilite a 			avaliação do trabalho face aos objetivos definidos. A defesa inclui a documentação finalizada e demonstração do que foi desenvolvido (implementação).
	\end{enumerate}
	
O TCC, do curso de Engenharia da Computação, corresponde a um total de quatro (04) cr\'editos, a serem efetivados mediante matrícula, sendo que dois (02) cr\'editos serão cumpridos na disciplina TCC-I e dois (02) cr\'editos na disciplina  TCC-II. 


\subsection{Do Professor da Disciplina}

<<<<<<< HEAD:trabalho/CAP2.TEX

\subsection{Do Professor da Disciplina}

=======
>>>>>>> develop:trabalho/CAP2.TEX
O TCC funciona como uma disciplina e, portanto, possui um professor responsável, o qual encaminha para avaliação os trabalhos finais (TCC-I e TCC-II) e controla as notas finais. 

Compete ao professor da disciplina:

	\begin{itemize}
		\item Estabelecer reunião inicial com os alunos e professores para expor as normas do TCC e dar-lhes ciência desse documento

		\item Divulgar e fazer cumprir as normas referentes ao TCC
		
		\item Divulgar o calendário referente às atividades do TCC

<<<<<<< HEAD:trabalho/CAP2.TEX
		\item Penalizar o aluno pelo não cumprimento dos prazos determinados no calendário de atividades do TCC; (0,2 pts, na média final, por dia de atraso na entrega 		da proposta e monografia)
=======
		\item Penalizar o aluno pelo não cumprimento dos prazos determinados no calendário de atividades do TCC; (0,2 pts, na m\'edia final, por dia de atraso na entrega 		da proposta e monografia)
>>>>>>> develop:trabalho/CAP2.TEX
		
		\item Divulgar aos alunos a relação de professores e suas respectivas linhas de pesquisa para exercerem a atividade de orientação
		
		\item Coordenar a formação das duplas orientador/orientando

		\intem Coordenar o processo de substituição orientador/orientando
		
		\item Coordenar o processo de constituição das bancas e definir o cronograma de apresentação dos trabalhos
		
		\item Encaminhar aos membros da banca o respectivo trabalho para avaliação
		
		\item Convidar, quando possível, um avaliador externo para integrar a banca
		
		\item Providenciar os documentos necessários ao processo de avaliação

		\item Providenciar as declarações aos professores participantes da banca, bem como do orientador
		
<<<<<<< HEAD:trabalho/CAP2.TEX
		\item Encaminhar uma cópia definitiva do trabalho (TCC-II) à Biblioteca
=======
		\item Encaminhar uma c\´opia definitiva do trabalho (TCC-II) \´a Biblioteca
>>>>>>> develop:trabalho/CAP2.TEX
		
	\end{itemize}


\subsection{Professor Orientador}

O professor orientador tem a função de ajudar o aluno no direcionamento do seu trabalho, sem, entretanto, desenvolver partes desse 
trabalho para o aluno. O orientador, apenas sugere caminhos que o aluno deverá seguir, acompanha seu trabalho, motivando e corrigindo eventuais erros.

Antes de apresentar o TCC-I ou TCC-II, o aluno deve submetê-lo previamente, \textbf{e obrigatoriamente}, à apreciação de seu orientador. 
Dado o aval do mesmo, a proposta poderá ser encaminhada e apresentada para avaliação.

Compete ao Professor Orientador:

	\begin{itemize}
		\item Informar ao professor da disciplina a linha de pesquisa que irá atuar
	
		\item Orientar a elaboração do Trabalho de Conclusão
	
<<<<<<< HEAD:trabalho/CAP2.TEX
		\item Auxiliar o aluno na resolução de problemas conceituais, técnicos e de relacionamento decorrentes da atividade
	
		\item Estabelecer o plano e cronograma de trabalho em conjunto com o orientando
	
		\item Informar o orientando sobre as normas, procedimentos e critérios de avaliação respectivos
=======
		\item Auxiliar o aluno na resolução de problemas conceituais, t\'ecnicos e de relacionamento decorrentes da atividade
	
		\item Estabelecer o plano e cronograma de trabalho em conjunto com o orientando
	
		\item Informar o orientando sobre as normas, procedimentos e crit\'erios de avaliação respectivos
>>>>>>> develop:trabalho/CAP2.TEX
	
		\item Informar ao aluno, caso haja atraso no cronograma de trabalho ou o não cumprimento das orientações, se o trabalho tem condições ou não de ser 
		encaminhado para avaliação
	
		\item liberar o trabalho para que haja a apresentação do aluno bem como informar ao professor da disciplina quanto à apresentação do aluno (Anexo C)
	
		\item Rubricar as 3 (três) vias encaminhadas para avaliação (TCC-I ou TCC-II) quando estiver ciente e de acordo, 
		conforme suas orientações, do material entregue
	
		\item Presidir a banca examinadora do trabalho por ele orientado
	
		\item Comunicar ao professor da disciplina situações que exijam providências, assim que ocorrerem.
	\end{itemize}


\subsection{Co-orientador}

	\begin{itemize}	
		\item Será solicitado formalmente pelo orientador a Professor da disciplina e será designado por esse para atender questão específica do trabalho
		
		\item Trabalhará em conjunto ao orientador e desempenhará papel solicitado pelo mesmo.
	\end{itemize}


\subsection{Orientando}

Compete ao Orientando:

	\begin{itemize}
		\item Comparecer às reuniões marcadas pelo professor da disciplina sobre o Trabalho de Conclusão
		
		\item Escolher a temática a ser trabalhada em consonância com as linhas de pesquisa do curso
		
		\item Contatar professor para definir orientador e informar a Coordenação do Projeto (entrega do Anexo-B)
		
		\item Cumprir as datas limites determinadas no calendário de atividades do TCC. O não cumprimento dos prazos será penalizado com perda de pontuação; 
<<<<<<< HEAD:trabalho/CAP2.TEX
		(0,2 pts, na média final, por dia de atraso na entrega da monografia)
=======
		(0,2 pts, na m\'edia final, por dia de atraso na entrega da monografia)
>>>>>>> develop:trabalho/CAP2.TEX
		
		\item Comparecer às orientações sobre o trabalho; o não comparecimento de três (03) orientações seguidas implica em reprovação por falta
		
		\item Seguir as orientações do professor designado à orientação
		
		\item Cumprir o plano e o cronograma de trabalho elaborado em conjunto com o professor-orientador
		
		\item Comunicar ao professor da disciplina toda e qualquer situação que possa comprometer, de alguma forma, o processo de elaboração, bem como, a conclusão do 			trabalho o quanto antes, para que a coordenação possa analisar o ocorrido e tomar as providências cabíveis
		
		\item Comparecer perante a banca na data, hora e local estabelecido para a realização da sessão de avaliação do TCC
	\end{itemize}


\subsection{Os Acompanhamentos de Orientação}

As reuniões de orientação deverão ser documentadas conforme modelo presente no Anexo A e serão entregues ao professor da disciplina no dia da entrega da carta (ANEXO A) solicitando defesa de TCC.
Tanto professor orientador como orientando deverão ter uma cópia dos acompanhamentos de orientação.
	

\subsection{A Banca Examinadora}

A banca examinadora do TCC-I e TCC –II, deverá ser composta por, no mínimo, 3 professores. A banca será constituída pelo 
professor orientador e por dois outros professores. Se houver co-orientação, o professor co-orientador pode compor a banca, contudo sua avaliação não computará nota para o alunosua avaliação não computará nota para o aluno 

<<<<<<< HEAD:trabalho/CAP2.TEX
Os membros da banca examinadora poderão sugerir alterações no trabalho (parte escrita e/ou implementação). Para o TCC-I as alterações deverão ser feitas, com o acompanhamento do orientador, para que sejam incluídas no trabalho e avaliadas no TCC-II. Para o TCC-II, estas deverão ser feitas até duas semanas depois da apresentação (ver data limite), supervisionadas pelo professor-orientador, para constar no(s) volume(s) final(is) do TCC, que ficará à disposição na biblioteca
=======
Os membros da banca examinadora poderão sugerir alterações no trabalho (parte escrita e/ou implementação). Para o TCC-I as alterações deverão ser feitas, com o acompanhamento do orientador, para que sejam incluídas no trabalho e avaliadas no TCC-II. Para o TCC-II, estas deverão ser feitas at\'e duas semanas depois da apresentação (ver data limite), supervisionadas pelo professor-orientador, para constar no(s) volume(s) final(is) do TCC, que ficará à disposição na biblioteca
>>>>>>> develop:trabalho/CAP2.TEX

O volume final para arquivamento (TCC-II) só será aceito pela coordenação de TCC se estiver validado pelo professor orientador, indicando sua concordância com o conteúdo do mesmo, e a assinatura do aluno


\subsection{Seminários}
Conforme calendário os seminários destinados aos alunos matriculados em TCC, abordam temas que auxiliarão na elaboração do documento escrito e na defesa.

Serão e seminários:
	\begin{itemize}
		\item Seminário I – Estrutura do Trabalho de Conclusão de Curso

		\item Seminário II – Normas ABNT

		\item Seminário III – Apresentação do TCC (defesa e material)
	\end{itemize}

<<<<<<< HEAD:trabalho/CAP2.TEX
A participação do aluno nos seminários é um dos critérios que consta na ata de avaliação. O não comparecimento acarretará perda de 0,2 pt por seminário.
=======
A participação do aluno nos seminários \'e um dos crit\'erios que consta na ata de avaliação. O não comparecimento acarretará perda de 0,2 pt por seminário.
>>>>>>> develop:trabalho/CAP2.TEX


\subsection{As Datas Limite}

As datas limites serão estabelecidas e divulgadas de acordo com o calendário acadêmico de cada período acadêmico.


\subsection{Nota Final}

Para aprovação do aluno no TCC, o mesmo deverá:

	\begin{itemize}
		\item Atender à exigência da frequência mínima de 75$'%'$ (setenta e cinco) às orientações. 
		A frequência do aluno será validada a partir do formulário de acompanhamento de reunião de orientação os quais devem ser preenchidos a cada acompanhamento, 			pelo orientador e pelo aluno

		\item Obter, no mínimo, grau 6,0 (seis). Este grau será composto pela m\'edia aritm\'etica das avaliações dos membros da banca examinadora. Cada membro da banca 			receberá uma planilha com itens a avaliar (por notas). Ao t\'ermino da defesa será preenchida uma ata final de avaliação de TCC constando a m\'edia final do aluno.
		Caso o aluno não alcance grau mínimo seis (6,0) deverá matricular-se novamente na disciplina para desenvolver novamente o trabalho ou concluir o 			desenvolvimento do mesmo
	\end{itemize}


\section{Trabalho de Conclusão de Curso I (TCC-I)}

O aluno, em parceria com um professor orientador, deve delimitar um tema, a ser abordado, dentro das linhas de pesquisa do curso.

Deve então dar início à documentação de seu trabalho elaborando uma monografia com os capítulos contendo a fundamentação teórica e modelagem da implementação do trabalho proposto. Ao final do semestre defendê-lo à uma banca examinadora.

\subsection{Estrutura do TCC-I}

Na monografia, o aluno deverá documentar seu trabalho para ser arquivado e, no futuro, referenciado por outras pessoas, lembrando sempre que Trabalho de Conclusão de Curso deve ser escrito tendo em vista uma metodologia científica. 

A monografia deve seguir a seguinte estrutura: 
	
	\indent \textbf{Capa} \\
	\indent \textbf{Folha de Rosto} \\
	\indent \textbf{Ficha para Catalogação} (deve ser impressa no verso da folha de rosto) \\
	\indent \textbf{Epígrafe} (opcional) \\
	\indent \textbf{Dedicatória} (opcional) \\
	\indent \textbf{Agradecimentos} (opcional) \\
	\indent \textbf{Resumo} \\
	\indent \textbf{Abstract} (resumo em inglês) \\
	\indent \textbf{Sumário} \\
	\indent \textbf{Lista de Figuras} (opcional) \\
	\indent \textbf{Lista de Tabelas} (opcional) \\
	\indent \textbf{Lista de Abreviaturas e siglas} \\
	\indent \textbf{Introdução} \\
	\indent \textbf{Desenvolvimento} \\
	\indent \textbf{Conclusão} \\
	\indent \textbf{Referências Bibliográficas} \\
	\indent \textbf{Obras Consultadas} \\
	\indent \textbf{Anexos e/ou Apêndices} (opcional) \\\\

\textbf{Introdução} - \'e o primeiro capítulo da monografia. Apresenta o contexto do trabalho proposto com a definição do problema, os objetivos (geral e específicos), os motivos que levaram à decisão de se abordar o tema e a organização do trabalho.

\textbf{Desenvolvimento} - corresponde aos demais capítulos da monografia, que descrevem sobre o tema proposto, revisão da literatura, metodologia aplicada, ferramentas e modelagem (se aplicável) do trabalho a ser implementado.

<<<<<<< HEAD:trabalho/CAP2.TEX
\textbf{Conclusão} - como o TCC-I é o início da monografia não será possível uma conclusão, portanto devem ser apresentadas as dificuldades encontradas, até o momento no trabalho, e resultados esperados do trabalho proposto.
=======
\textbf{Conclusão} - como o TCC-I \'e o início da monografia não será possível uma conclusão, portanto devem ser apresentadas as dificuldades encontradas, at\'e o momento no trabalho, e resultados esperados do trabalho proposto.
>>>>>>> develop:trabalho/CAP2.TEX

A formatação (margens, espaçamentos, citações, paginação, etc.) de todo o documento, deve estar voltada para um trabalho científico, portanto, 
os alunos devem seguir o modelo de monografia adotado pelo curso e disponível no site do mesmo. 
Solicitamos ainda aos alunos que utilizem as obras abaixo:

	\begin{itemize}	
<<<<<<< HEAD:trabalho/CAP2.TEX
		\item  FURASTÉ, Pedro Augusto. Normas Técnicas para o Trabalho Científico (Nova ABNT). 14ª edição. Porto Alegre, 2006.
=======
		\item  FURAST\'e, Pedro Augusto. Normas T\'ecnicas para o Trabalho Científico (Nova ABNT). 14ª edição. Porto Alegre, 2006.
>>>>>>> develop:trabalho/CAP2.TEX

		\item SILVA, Edna Lúcia da. Metodologia da Pesquisa e Elaboração de Dissertação – 3ª ed. rev. e atual. – Florianópolis: Laboratório de Ensino a Distância da 			UFSC, 2001.

		\item BRASIL, ABNT – Associação Brasileira de Normas T\'ecnicas. NBR 14724.

		\item BRASIL, ABNT – Associação Brasileira de Normas T\'ecnicas. NBR 10520.

<<<<<<< HEAD:trabalho/CAP2.TEX
		\item BRASIL, ABNT – Associação Brasileira de Normas Técnicas. NBR 6023.
=======
		\item BRASIL, ABNT – Associação Brasileira de Normas T\'ecnicas. NBR 6023.
>>>>>>> develop:trabalho/CAP2.TEX
	\end{itemize}
	

\subsection{Avalição do TCC-I}

<<<<<<< HEAD:trabalho/CAP2.TEX
O TCC-I deverá ser apresentado perante uma banca examinadora a ser definida pelo professor da disciplina, para a qual o aluno apresentará seu trabalho, desde a justificativa do problema que o levou a desenvolvê-lo até as discussões do material levantado.
=======
O TCC-I deverá ser apresentado perante uma banca examinadora a ser definida pelo professor da disciplina, para a qual o aluno apresentará seu trabalho, desde a justificativa do problema que o levou a desenvolvê-lo at\'e as discussões do material levantado.
>>>>>>> develop:trabalho/CAP2.TEX

O aluno terá 25 (vinte e cinco) minutos para defesa de sua proposta, onde utilizará os recursos audiovisuais que achar necessário e serão utilizados mais 10 (dez) minutos para responder aos questionamentos de cada membro da banca avaliadora. A banca será constituída pelo professor orientador (presidente) e por dois outros professores, podendo ser um convidado externo. Avaliado o trabalho escrito e “ouvidas” as sugestões da banca, o aluno deverá fazer as modificações necessárias.

No TCC-I caso o aluno não alcance grau mínimo 6,0 (seis) deverá matricular-se novamente na disciplina pra desenvolver novamente o trabalho (ou concluir o desenvolvimento do mesmo)  em TCC-I.
<<<<<<< HEAD:trabalho/CAP2.TEX
É necessário que o aluno seja aprovado em TCC-I para a conclusão do trabalho em TCC-II. 
=======
\'e necessário que o aluno seja aprovado em TCC-I para a conclusão do trabalho em TCC-II. 
>>>>>>> develop:trabalho/CAP2.TEX


\section{Trabalho de Conclusão de Curso II (TCC-II)}

<<<<<<< HEAD:trabalho/CAP2.TEX
O aluno deverá por em prática a modelagem apresentada em TCC-I além de concluir a monografia (implementação, resultados obtidos, etc.). Haverá nova defesa da documentação e demonstração do que foi desenvolvido (implementação) de acordo com esta documentação.
=======
O aluno deverá por em prática a modelagem apresentada em TCC-I al\'em de concluir a monografia (implementação, resultados obtidos, etc.). Haverá nova defesa da documentação e demonstração do que foi desenvolvido (implementação) de acordo com esta documentação.
>>>>>>> develop:trabalho/CAP2.TEX


\subsection{Estrutura do TCC-II}

Apresenta a mesma estrutura do TCC-I. Entretanto, no TCC-II, o aluno irá complementar a monografia de acordo com as solicitações feitas pela banca examinadora (na defesa do TCC-I) e com tópicos relacionados à sua implementação. 

A monografia deve seguir a seguinte estrutura: 

	\indent \textbf{Capa} \\
	\indent \textbf{Folha de Rosto} \\
	\indent \textbf{Ficha para Catalogação} (deve ser impressa no verso da folha de rosto) \\
	\indent \textbf{Epígrafe} (opcional) \\
	\indent \textbf{Dedicatória} (opcional) \\
	\indent \textbf{Agradecimentos} (opcional) \\
	\indent \textbf{Resumo} \\
	\indent \textbf{Abstract} (resumo em inglês) \\
	\indent \textbf{Sumário} \\
	\indent \textbf{Lista de Figuras} (opcional) \\
	\indent \textbf{Lista de Tabelas} (opcional) \\
	\indent \textbf{Lista de Abreviaturas e siglas} \\
	\indent \textbf{Introdução} \\
	\indent \textbf{Desenvolvimento} \\
	\indent \textbf{Conclusão} \\
	\indent \textbf{Referências Bibliográficas} \\
	\indent \textbf{Obras Consultadas} \\
	\indent \textbf{Anexos e/ou Apêndices} (opcional) \\\\


\textbf{Introdução} - \'e o primeiro capítulo da monografia. Apresenta o contexto do trabalho proposto com a definição do problema, os objetivos (geral e específicos), os motivos que levaram à decisão de se abordar o tema e a organização do trabalho.

\textbf{Desenvolvimento} - corresponde aos demais capítulos da monografia, que descrevem sobre o tema proposto, revisão da literatura, metodologia aplicada, ferramentas e modelagem (se aplicável) do trabalho a ser implementado.

\textbf{Conclusão} - se os objetivos foram atingidos, dificuldades encontradas e sugestões para trabalhos futuros.

A formatação (margens, espaçamentos, citações, paginação, etc.) de todo o documento, deve estar voltada para um trabalho científico, portanto, 
os alunos devem seguir o modelo de monografia adotado pelo curso e disponível no site do mesmo. 
Solicitamos ainda aos alunos que utilizem as obras abaixo:

	\begin{itemize}	
<<<<<<< HEAD:trabalho/CAP2.TEX
		\item  FURASTÉ, Pedro Augusto. Normas Técnicas para o Trabalho Científico (Nova ABNT). 14ª edição. Porto Alegre, 2006.
=======
		\item  FURAST\'e, Pedro Augusto. Normas T\'ecnicas para o Trabalho Científico (Nova ABNT). 14ª edição. Porto Alegre, 2006.
>>>>>>> develop:trabalho/CAP2.TEX

		\item SILVA, Edna Lúcia da. Metodologia da Pesquisa e Elaboração de Dissertação – 3ª ed. rev. e atual. – Florianópolis: Laboratório de Ensino a Distância da 			UFSC, 2001.

		\item BRASIL, ABNT – Associação Brasileira de Normas T\'ecnicas. NBR 14724.

		\item BRASIL, ABNT – Associação Brasileira de Normas T\'ecnicas. NBR 10520.

		\item BRASIL, ABNT – Associação Brasileira de Normas T\'ecnicas. NBR 6023.
	\end{itemize}


\subsection{Avaliação do TCC-II}

O TCC-II deverá ser apresentado perante uma banca examinadora a ser definida pelo professor da disciplina, para a qual o aluno apresentará seu trabalho, desde a justificativa do problema que o levou a desenvolvê-lo at\'e as discussões do material levantado e a conclusão.

Esta apresentação deverá ser, necessariamente, oral e descritiva, onde o aluno deverá tamb\'em na parte oral resumir as principais funções do sistema, o modo como será usado na organização ou ambiente e os seus benefícios.

Para esta apresentação oral, o aluno deverá preparar o que irá falar e utilizar recursos didáticos, considerando o tempo de 45 minutos; cada membro da banca avaliadora terá 10 minutos para questionamentos.

Não serão aceitas justificativas para a não demonstração das implementações, implicando assim em reprovação.

As apresentações dos TCCs são abertas ao público interessado. Sugere-se, fortemente, que os alunos de TCC-I assistam às bancas de seus colegas de TCC-II, como experiência.

Caso o aluno não alcance a nota mínima de 6,0 pontos em TCC-II, deverá matricular-se, novamente na disciplina, no próximo semestre.
	
A nota do aluno só será lançada mediante entrega de 2 cópias da versão revisada com visto do orientador, já incluindo as modificações sugeridas pela banca, no formato final (impresso em jato de tinta ou laser), em capa dura, na cor preta, bem como um CD com a versão final do trabalho e os produtos resultantes da pesquisa, quando for o caso.

	


%% T�tulo do Capitulo 3
\chapter{T�tulo do Terceiro Cap�tulo}
\label{cap:titulo do terceiro capitulo}

Lorem ipsum dolor sit amet, consectetur adipiscing elit. Quisque turpis risus, pulvinar quis pellentesque vel, sodales a lorem. Nulla eget ante a odio pretium consequat id at odio. Maecenas mauris dolor, laoreet eu consequat sed, mollis sed sapien. Integer dolor nisl, malesuada consectetur vestibulum id, faucibus ac elit. Donec leo justo, consectetur ut suscipit vel, rutrum ac mauris. Sed ut augue tellus. Vestibulum ante ipsum primis in faucibus orci luctus et ultrices posuere cubilia Curae; Integer ac diam sed urna tristique lobortis sit amet non justo. Pellentesque laoreet ipsum neque. Nullam sit amet lorem purus. Cras in placerat dolor.

Phasellus fringilla nulla eget nunc adipiscing in volutpat enim bibendum. Aliquam et ante at ipsum molestie sodales. Pellentesque mattis venenatis metus, at tristique diam ullamcorper a. Nulla non risus et libero accumsan facilisis id ac justo. Ut eleifend placerat velit quis vehicula. Donec a sapien justo, pellentesque ultricies neque. Aenean consequat, tortor ac porta porta, nisl ante condimentum magna, non dictum est sem vel sapien. Vivamus vel arcu a nunc lacinia volutpat et dignissim tellus. Aenean augue massa, ultricies ut blandit non, porta a risus. Aenean dapibus, turpis sit amet eleifend laoreet, nulla erat volutpat est, ut pulvinar metus sem nec erat. Phasellus ullamcorper ante eu est venenatis at egestas nulla mollis. Donec libero lacus, venenatis ac suscipit ac, ultricies posuere sapien. Nam fermentum dapibus pulvinar. Aenean non libero quis orci semper pulvinar. Pellentesque sed augue vitae tellus molestie vulputate.

Mauris quis nunc justo, porttitor eleifend metus. Pellentesque at turpis at ante consequat lacinia aliquam sit amet erat. Duis nec dolor mauris. Praesent tristique neque vitae diam cursus adipiscing. Pellentesque malesuada velit sed magna convallis sodales ac vitae libero. Curabitur sagittis ullamcorper est ac mollis. Aenean metus lectus, iaculis id tincidunt quis, tincidunt ut dolor. Integer porttitor tincidunt augue sed condimentum. Mauris pellentesque vestibulum justo, vel pellentesque tellus suscipit in. Nulla eget sem augue. Vestibulum in sapien nec nibh accumsan ultricies. Aliquam varius consectetur lorem sed luctus. Nullam id ipsum ut ante tristique tempor. Donec vel arcu turpis, in volutpat magna. In hac habitasse platea dictumst. Phasellus eget sagittis neque. Suspendisse dictum ornare sapien, ut ullamcorper magna pharetra faucibus.


\begin{table}[!h]
\begin{tabular}{|r|c|c|}
\hline
& Chessboard top view & Chessboard perspective view\\
\hline
Selection whit side moviments & 6.02\stackrel{+}{-}5.22 & 7.01\stackrel{+}{-}6.84\\
\hline
Selection whith indepth moviments & 6.29\stackrel{+}{-}4.99 & 12.22\stackrel{+}{-}11.33\\
\hline
Manipulation with side moviments & 4.66\stackrel{+}{-}4.94 & 3.47\stackrel{+}{-}2.20\\
\hline
Manipulation with indepth moviments & 5.71\stackrel{+}{-}4.55 & 5.37\stackrel{+}{-}3.28\\
\hline
\end{tabular}
\caption{Vari�veis consider�veis na avalia��o de t\'ecnicas de intera��o}
\end{table}


\\

\begin{table}
 \begin{center}
  \begin{tabular}{|l||c|c|c|c|}
    \hline
    \multicolumn{5}{|c|}{2003} \\ \hline \hline
    \multicolumn{1}{|c||}{Fase} & Mar�o     & Abril     & Maio     & Junho   \\ \hline
    1    & $\bullet$ &           &          &         \\
    2    &           & $\bullet$ & $\bullet$&         \\ 
    3    &           &           & $\bullet$& $\bullet$ \\
    \hline
    \hline
   \end{tabular}
   \caption{Exemplo de cronograma usando \textit{bullet}}
  \end{center}
 \label{tab3}
\end{table}


Curabitur aliquam purus vitae velit elementum sollicitudin eget at turpis. Mauris nec mauris ac tortor elementum ultrices. Ut nec nisl arcu. Nullam non nunc ante. Donec ac mauris ut nulla vulputate lobortis. In ac quam tellus. Vivamus viverra tortor quis lectus dictum malesuada. Sed posuere, nunc sed aliquam pharetra, felis massa venenatis justo, sed mattis massa magna quis quam. Donec et velit ac orci aliquet blandit. Integer sit amet dolor ac sem pretium tincidunt. Nam volutpat convallis elementum. Ut accumsan arcu et ipsum accumsan tincidunt.

Cras nec quam mi, ut mattis ante. Lorem ipsum dolor sit amet, consectetur adipiscing elit. Sed fringilla auctor dictum. Nam hendrerit sapien sed massa consequat rutrum. Nullam congue, augue sed commodo malesuada, lectus nulla mollis magna, eget semper risus nisl eget elit. Duis vitae hendrerit massa. In a odio nunc, sit amet mollis dolor. In accumsan suscipit dui, a vestibulum diam condimentum ullamcorper. Etiam ut quam arcu, ac tristique ante. Vestibulum imperdiet elit non ante tristique accumsan. Donec vulputate fringilla tempor. Proin porttitor nisi nisi. Fusce vel ullamcorper orci. Lorem ipsum dolor sit amet, consectetur adipiscing elit. 


\begin{figure}[!htp]
\centering
\includegraphics[!htp]{Fig_6_8Holt.jpg}
\caption{Uma t�pica figura}
\end{figure}


Donec arcu erat, eleifend in volutpat a, sodales ut leo. Nam vulputate ornare nisl, eget lobortis magna posuere a. Praesent sit amet est est, sit amet sollicitudin nibh. Nam sagittis nulla fermentum elit aliquet aliquet. Proin at sem vitae velit interdum viverra nec at diam. Aenean arcu dui, eleifend ut gravida ac, ornare placerat urna. Suspendisse aliquam condimentum metus nec faucibus. Lorem ipsum dolor sit amet, consectetur adipiscing elit. Aliquam non leo ipsum, sollicitudin consectetur mi. Duis erat eros, tempor in consequat in, scelerisque eget ante. Praesent placerat vulputate rhoncus. Ut et ante risus, et molestie felis.


\begin{figure}[!htp]
\centering
\includegraphics[scale = .5]{graficos-online-gratis.png}
\caption{Uma t�pica figura}
\end{figure}


Vivamus ultricies tincidunt lacus ut pharetra. Sed fringilla hendrerit tempus. Suspendisse potenti. Cras hendrerit tortor ac est condimentum pellentesque. Morbi pretium lectus nec sapien laoreet eu malesuada diam adipiscing. Aliquam nisl ipsum, fermentum ut aliquam nec, varius sit amet nisi. Pellentesque interdum cursus malesuada. Vestibulum ante ipsum primis in faucibus orci luctus et ultrices posuere cubilia Curae; Nullam malesuada bibendum tortor, ut bibendum lorem varius eu. In eros orci, volutpat ut facilisis sit amet, commodo quis nulla.


\begin{figure}[!htp]
\centering
\includegraphics[scale = .5]{graf-fig1.png}
\caption{Uma t�pica figura}
\end{figure}



Sed lectus metus, mollis nec vulputate id, imperdiet eget urna. Nam ut dolor at metus venenatis suscipit et in ligula. In hac habitasse platea dictumst. Mauris scelerisque dolor sed nisl mattis accumsan. Aliquam vulputate placerat feugiat. Pellentesque faucibus neque mi. Etiam porttitor varius tempus. Mauris varius porttitor posuere. Pellentesque iaculis imperdiet lobortis. Sed vulputate purus nec felis rutrum molestie. 


\begin{figure}[!htp]
\centering
\includegraphics[!htp]{grafico}
\caption{Uma t�pica figura}
\end{figure}




%% T�tulo do Capitulo 4
\input{CAP4.TEX}

% incluir referencias
\bibliography{tese}
\bibliographystyle{apa}


%\include{anexos}

\end{document}
