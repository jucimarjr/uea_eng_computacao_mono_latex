
Phasellus fringilla nulla eget nunc adipiscing in volutpat enim bibendum. 
Aliquam et ante at ipsum molestie sodales. Pellentesque mattis venenatis metus, 
at tristique diam ullamcorper a. Nulla non risus et libero accumsan facilisis 
id ac justo. Ut eleifend placerat velit quis vehicula. 

Neste contexto, a metodologia proposta, que envolve a verifica\c{c}\~{a}o formal
do mecanismo de reconhecimento de ondas eletrocardiogr\'aficas e
sequ\^{e}ncias de ondas com a utiliza\c{c}\~{a}o da ferramenta de verifica\c{c}\~{a}o
de modelos {\em Verus}, tem apresentado resultados concretos na
corre\c{c}\~{a}o e projeto deste mecanismo de reconhecimento atrav\'es da
identifica\c{c}\~{a}o e corre\c{c}\~{a}o de falhas, tornando o sistema mais seguro e
confi\'avel.

Aenean metus lectus, iaculis id tincidunt quis, 
tincidunt ut dolor. Integer porttitor tincidunt augue sed condimentum. Mauris pellentesque 
vestibulum justo, vel pellentesque tellus suscipit in. Nulla eget sem augue. 
Vestibulum in sapien nec nibh accumsan ultricies. Aliquam varius consectetur lorem sed luctus. 
Nullam id ipsum ut ante tristique tempor.

A aplica\c{c}\~ao da metodologia permitiu que outras ondas fossem
reconhecidas pelo mecanismo de leitura de dados
eletrocardiogr\'aficos, de forma que fosse feito o
re-projeto do autômato para reconhecimento de ondas, bem como
fosse projetado um aut\^omato para o reconhecimento de sequ\^encias
de ondas, contribuindo de maneira concreta para o desenvolvimento
da aplica\c{c}\~{a}o biom\'edica.


Palavras Chave: Lorem, Ipsum
