
Phasellus fringilla nulla eget nunc adipiscing in volutpat enim bibendum. 
Aliquam et ante at ipsum molestie sodales. Pellentesque mattis venenatis metus, 
at tristique diam ullamcorper a. Nulla non risus et libero accumsan facilisis 
id ac justo. Ut eleifend placerat velit quis vehicula. 

Neste contexto, a metodologia proposta, que envolve a verificação formal
do mecanismo de reconhecimento de ondas eletrocardiográficas e
seqüências de ondas com a utilização da ferramenta de verificação
de modelos {\em Verus}, tem apresentado resultados concretos na
correção e projeto deste mecanismo de reconhecimento através da
identificação e correção de falhas, tornando o sistema mais seguro e
confiável.

Aenean metus lectus, iaculis id tincidunt quis, 
tincidunt ut dolor. Integer porttitor tincidunt augue sed condimentum. Mauris pellentesque 
vestibulum justo, vel pellentesque tellus suscipit in. Nulla eget sem augue. 
Vestibulum in sapien nec nibh accumsan ultricies. Aliquam varius consectetur lorem sed luctus. 
Nullam id ipsum ut ante tristique tempor.

A aplicação da metodologia permitiu que outras ondas fossem
reconhecidas pelo mecanismo de leitura de dados
eletrocardiográficos, de forma que fosse feito o
re-projeto do autômato para reconhecimento de ondas, bem como
fosse projetado um autômato para o reconhecimento de seqüências
de ondas, contribuindo de maneira concreta para o desenvolvimento
da aplicação biomédica.


Palavras Chave: Lorem, Ipsum
