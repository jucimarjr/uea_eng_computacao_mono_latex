\chapter {Como Escrever Uma monografia} \label{cap:comoEscrever}

\section{Ao Candidato}

O texto abaixo foi livremente adaptado de \cite{COMER2010} para ajudar os alunos a escreverem suas monografias.
Se você est\'{a} se preparando para escrever uma monografia em uma \'{a}rea experimental da Engenharia da Computa\c{c}\~{a}o. 
A menos que você tenha escrito muitos documentos formais antes, você tem uma surpresa: isso \'{e} dif\'{i}cil!

Existem dois caminhos poss\'{i}veis para o sucesso:

\begin{itemize}
	\item Planejamento
	
	Poucas pessoas pegam esse caminho. As poucas que pegam, deixam a universidade t\~{a}o r\'{a}pido, 
	que eles malmente s\~{a}o notados. Se você quer fazer a impress\~{a}o final e ter uma longa carreira como um 
	estudante de gradua\c{c}\~{a}o, n\~{a}o escolha este caminho.
	
	\item Perseveran\c{c}a
	
	Tudo que você tem que fazer \'{e} sobreviver à sua banca julgadora. 
	A boa not\'{i}cia \'{e} que eles s\~{a}o bem mais velhos que você, assim você pode adivinhar quem vai eventualmente 
	``expirar''  primeiro (morrer).
	A m\'{a} not\'{i}cia \'{e} que \'{e} que eles s\~{a}o mais experientes nesse jogo (afinal, eles perseveraram na frente da banca deles, n\~{a}o!?).
	
\end{itemize}
	
Aqui est\~{a}o algumas linhas-guia que podem ajud\'{a}-lo quando você finalmente levar a s\'{e}rio escrever. 
A lista segue infinitamente; você provavelmente n\~{a}o vai querer ler isso tudo de uma vez. Mas, por favor, leia isso antes de 
escrever qualquer coisa.

\hfill

\section{A Id\'{e}ia Geral}	

\begin{enumerate}
	
	\item Uma monografia \'{e} um documento formal onde o aluno descreve a realiza\c{c}\~{a}o de um trabalho t\'{e}cnico onde usou as t\'{e}cnicas e conceitos aprendidos durante o curso de 	gradua\c{c}\~{a}o.

	\item Em geral, toda afirma\c{c}\~{a}o em uma monografia deve ser embasada ou por uma referência em literatura cient\'{i}fica publicada ou por um trabalho original.
	Acima de tudo, uma monografia n\~{a}o repete os detalhes do pensamento cr\'{i}tico e an\'{a}lises encontradas nas fontes publicadas; usa o resultado como fato e 
	referencia o leitor às fontes para mais detalhes. 

	\item Cada senten\c{c}a em uma monografia deve ser completa e correta gramaticalmente. Al\'{e}m do mais, a monografia deve satisfazer estritamente as 
	regras da gram\'{a}tica formal (e.x., sem contra\c{c}\~{o}es, sem coloquialismo, sem pron\'{u}ncias erradas, sem jarg\~{a}o t\'{e}cnico indefinido, sem piadas escondidas e sem g\'{i}rias,
	mesmo quando tais termos ou frases s\~{a}o de comum uso na l\'{i}ngua falada). Realmente, a escrita de uma monografia deve ser um cristal limpo.
	Sombras de significados importam; a terminologia e a prosa devem fazer uma fina distin\c{c}\~{a}o. As palavras devem carregar exatamente o sentido pretendido, nada mais e nada 	menos.

	\item Cada afirma\c{c}\~{a}o em uma monografia deve ser correta e defens\'{i}vel no sentido l\'{o}gico e cient\'{i}fico. Acima de tudo, as discuss\~{o}es em uma monografia devem satisfazer
	a maioria das estritas regras de l\'{o}gica aplicada à matem\'{a}tica e Engenharia.                       

\end{enumerate}

\section{O que se Deve Aprender do Exerc\'{i}cio}

\begin{enumerate}

	\item Todo engenheiro precisa comunicar descobertas; a monografia fornece um treinamento para comunica\c{c}\~{a}o com outros engenheiros.

	\item Escrever uma monografia requer que o estudante pense profundamente, para organizar a discuss\~{a}o t\'{e}cnica, para reunir argumentos que convencer\~{a}o outros engenheiro,
	e seguir as regras para uma rigorosa apresenta\c{c}\~{a}o dos argumentos e discuss\~{o}es.

\end{enumerate}

\section{Regra do Polegar}

Boa escrita \'{e} essencial para uma monografia. Entretanto, boa escrita n\~{a}o pode compensar uma escassez de id\'{e}ias ou conceitos.
Pelo contr\'{a}rio, uma apresenta\c{c}\~{a}o limpa sempre exp\~{o}e fraquezas.

\section{Defini\c{c}\~{o}es e Terminologia}

\begin{enumerate}

	\item Cada termo t\'{e}cnico usado em uma monografia deve ser definido ou por uma referência à uma defini\c{c}\~{a}o publicada anteriormente (para termos padr\~{o}es com seus significados usuais)
	ou por uma precisa, n\~{a}o-amb\'{i}gua defini\c{c}\~{a}o que aparece antes do termo ser usado (para termos novos ou um termo padr\~{a}o usado de maneira n\~{a}o usual).

	\item Cada termo deve ser usado de uma e \'{u}nica maneira por toda monografia.			

	\item A forma mais f\'{a}cil de evitar uma longa s\'{e}rie de defini\c{c}\~{o}es \'{e} incluir uma afirma\c{c}\~{a}o: ``a terminologia usada no decorrer deste documento segue a mesma dada em [CITA\c{c}\~{a}o].''
	Ent\~{a}o, s\'{o} defina exce\c{c}\~{o}es.

	\item O cap\'{i}tulo introdut\'{o}rio pode dar o intuito (i.e., defini\c{c}\~{o}es informais) dos termos fornecidos, os quais ser\~{a}o mais precisamente definidos depois.

\end{enumerate}

\section{Termos e Frase a Evitar}
	
\begin{itemize}

	\item Adv\'{e}rbios

		\indent Na maioria das vezes, s\~{a}o ``muito frequentemente usados demais''. Ao inv\'{e}s deles use palavras mais fortes. Algu\'{e}m pode dizer, por exemplo, `` Escritores abusam de adv\'{e}rbios.''
	
	\item Piadas ou Trocadilhos

		\indent Esses n\~{a}o têm lugar em um documento formal.
	
	\item ``Ruim'', ``Bom'', ``Terr\'{i}vel'', ``Est\'{u}pido''

		\indent Uma monografia n\~{a}o faz julgamento moral. Use ``incorreto/correto'' para se referir à erros ou corretudes de fato.
		Use palavras precisas ou frases para avaliar qualidade (e.x. ``m\'{e}todo A requer menos recurso computacional que m\'{e}todo B''). Em geral, deve-se evitar todos os julgamentos qualitativos.
		
	\item ``Verdade'', ``Puro''

		\indent No mesmo senso de ``bom'' (\'{e} um julgamento).
		
	\item ``Perfeito''

		\indent Nada \'{e}.
		
	\item ``Uma solu\c{c}\~{a}o ideal''

		\indent Você est\'{a} julgando de novo.
		
	\item ``Hoje'', ``Tempos modernos''

		\indent Hoje \'{e} o ontem de amanh\~{a}.
		
	\item ``Logo''

		\indent Logo quanto? Hoje à noite? Pr\'{o}xima d\'{e}cada?
		
	\item ``Est\'{a}vamos surpresos ao ver ...''

		\indent Mesmo se você estivesse, e da\'{i}?
		
	\item ``Parece'', ``Aparentemente''

		\indent N\~{a}o importa como algo aparenta.
	
	\item ``Parece mostrar''

		\indent Tudo o que importa s\~{a}o os fatos.
		
	\item ``Em termos de''

		\indent Normalmente vago.
		
	\item ``Baseado em'', ``X-baseado'', ``Como base de''

		\indent Cuidado, pode ser vago.
		
	\item ``Diferente''

		\indent	N\~{a}o significa ``v\'{a}rios''. Diferente do que?
		
	\item ``Na luz de''

		\indent	Coloquial. 
		
	\item ``Um monte de''

		\indent	Vago \& Coloquial
		
	\item ``Tipo de''

		\indent	Vago \& Coloquial
		
	\item ``Algo como''

		\indent	Vago \& Coloquial
		
	\item ``Mais ou Menos'' 

		\indent	Vago \& Coloquial
		
	\item ``N\'{u}mero de''

		\indent	Vago, você quer dizer, ``alguns'', ``muitos'' ou ``a maioria''? Uma afirma\c{c}\~{a}o quantitativa \'{e} prefer\'{i}vel.
		
	\item ``Devido a''

		\indent	Coloquial
		
	\item ``Provavelmente''

		\indent	Apenas se você souber a probabilidade estat\'{i}stica (se você sabe, afirme quantitativamente).
		
	\item ``Obviamente'', ``Claramente''

		\indent Tenha cuidado: \'{o}bvio/Claro para todos?
		
	\item ``Simples''

		\indent Pode ter uma conota\c{c}\~{a}o negativa, como em ``simpl\'{o}rio''.
		
	\item ``Junto com''

		\indent Use somente ``com''.
		
	\item ``Na verdade'', ``Realmente''

		\indent Defina os termos claramente para eliminar a necessidade de esclarecimento.
		
	\item ``O fato de''

		\indent Faz uma meta-senten\c{c}a; reformule a frase.
		
	\item ``Isso'', ``Aquilo''

		\indent Como em ``Estas causa envolvem.'' Raz\~{a}o: ``Isso'' pode referir ao sujeito da senten\c{c}a anterior, à toda senten\c{c}a anterior, todo o par\'{a}grafo anterior, toda a se\c{c}\~{a}o anterior, etc. 
		Mais importante, pode ser interpretado no sentido correto ou no meta-sentido. Por exemplo: \textit{``X faz Y. Isso significa...''} o leitor pode assumir ``isso'' referindo ao 			
		\textit	{Y} ou ao fato de \textit{X faz Y}. Mesmo quando restrito (e.x., ``esta computa\c{c}\~{a}o''), a frase \'{e} fraca e frequentemente amb\'{i}gua.
		
	\item ``Você ir\'{a} ler sobre isso''

		\indent A segunda pessoa n\~{a}o tem lugar em uma monografia.
		
	\item ``Eu vou descrever''

		\indent A primeira pessoa n\~{a}o tem lugar em uma monografia formal. Se auto-referência \'{e} essencial, escreva como ``Se\c{c}\~{a}o 10 descreve...''
		
	\item ``N\'{o}s'' como em ``Vemos que''

		\indent Uma armadilha a evitar. Raz\~{a}o: Quase toda senten\c{c}a pode ser escrita para come\c{c}ar com ``n\'{o}s'' porque `n\'{o}s'' pode se referir: ao leitor e autor, ao autor e consultor, ao autor e grupo de pesquisa, engenheiros de computa\c{c}\~{a}o, a toda comunidade de Engenharia da computa\c{c}\~{a}o, ou algum outro grupo n\~{a}o especificado. 

	\item ``Esperan\c{c}osamente, o programa''

		\indent Programas n\~{a}o tem esperan\c{c}a, n\~{a}o at\'{e} serem implementados com sistemas de IA. Ali\'{a}s, se você estiver escrevendo uma tese de IA, fale com outra pessoa: pessoas de IA tem seus pr\'{o}prios sistemas de regras.
		
	\item ``...um famoso pesquisador...''

		\indent N\~{a}o importa quem disse ou fez. De fato, tais afirma\c{c}\~{o}es prejudicam o leitor.
		
	\item Tenha cuidado usando: ``poucos, maioria, todos, algum, cada'' 

		\indent Uma monografia \'{e} precisa. Se a senten\c{c}a diz ``Maioria dos sistemas computacionais cont\'{e}m X'', você deve ser capaz de defender isso. Você tem certeza que conhece os fatos? Quantos computadores foram constru\'{i}dos e vendidos ontem?
		
	\item ``Deve'', ``Sempre''

		\indent Absolutamente?
		
	\item ``Deveria''

		\indent Quem disse isso?
		
	\item ``Prova'', ``Comprova''

		\indent Um matem\'{a}tico aceitaria que isso \'{e} uma prova?

	\item ``Pode'', ``Poderia''

		\indent Sua m\~{a}e provavelmente lhe disse a diferen\c{c}a.

\end{itemize}

\section{Voz}

Use constru\c{c}\~{o}es ativas. Por exemplo, diga ``o sistema operacional inicia o dispositivo'' ao inv\'{e}s de ``o dispositivo \'{e} iniciado pelo sistema operacional.''

\section{Tempo Verbal}

Escreva no presente. ``O sistema escreve a p\'{a}gina no disco e ent\~{a}o usa o frame...'' ao inv\'{e}s de ``O sistema usar\'{a} o frame depois de ter escrito a p\'{a}gina no disco...''

\section{Defina Nega\c{c}\~{o}es com Antecedência}

Exemplo: diga ``Nenhum bloco de dados espera na fila de sa\'{i}da'' ao inv\'{e}s de ``Um bloco de dados esperando sa\'{i}da n\~{a}o est\'{a} na fila.''

\section{Gram\'{a}tica e L\'{o}gica}

Tenha cuidado pois o sujeito de cada senten\c{c}a realmente faz o que o verbo diz q ele faz. 
Dizer ``Programas devem fazer chamada de processo usando a instru\c{c}\~{a}o X'' n\~{a}o \'{e} o mesmo que dizer ``Programas devem usar a instru\c{c}\~{a}o X quando chamam um procedimento.'' 
De fato, a primeira \'{e} evidentemente falsa! 
Outro exemplo: ``RPC requer programas para transmitir pacotes grandes'' n\~{a}o \'{e} o mesmo que ``RPC requer um mecanismo que permita programas transmitirem pacotes grandes.''

\section{Foco nos Resultados e n\~{a}o nas Pessoas/Circunst\^{a}ncias em que Foram Obtidos}
		
``Depois de trabalhar oito horas no laborat\'{o}rio naquela noite. n\'{o}s percebemos...'' n\~{a}o tem lugar na monografia. 
N\~{a}o importa quando você percebeu isso, ou quanto tempo você trabalhou para obter a resposta. 
Outro exemplo: ``Jim e eu chegamos aos n\'{u}meros mostrados na tabela 3 medindo...'' Ponha um agradecimento para Jim na monografia, mas n\~{a}o inclua nomes (nem mesmo o seu) no corpo principal. 
Você pode estar tentado a documentar uma longa s\'{e}rie de experimentos que n\~{a}o produziram nada ou uma coincidência que resultou em sucesso. Evite completamente isso. 
Em particular, n\~{a}o documente aparentemente influências m\'{i}sticas (e.x., ``se aquele gato n\~{a}o tivesse rastejado pelo buraco no ch\~{a}o, poder\'{i}amos 
n\~{a}o ter descoberto o indicador de erro do fornecimento de energia na ponte de rede''). Nunca atribua tais eventos à causas m\'{i}sticas ou dê a 
entender que for\c{c}as estranhas podem ter afetado seu resultado. 
Resumo: Prenda-se nos fatos evidentes. Descreva os resultados sem mencionar suas rea\c{c}\~{o}es ou eventos que o ajudaram a alcan\c{c}\'{a}-los.

\section{Evite Auto-Avalia\c{c}\~{a}o (Elogio e Cr\'{i}tica)}

Ambos os exemplos a seguir est\~{a}o incorretos: ``O m\'{e}todo esbo\c{c}ado na Se\c{c}\~{a}o 2 representa o maior avan\c{c}o em design de sistemas distribu\'{i}dos porque...'' 
``Embora a t\'{e}cnica na pr\'{o}xima se\c{c}\~{a}o n\~{a}o seja extraordin\'{a}rio,...''

\section{Referências à Trabalhos}

Sempre cita-se o artigo, n\~{a}o o autor. Assim, usa-se um verbo no singular para referir ao artigo, mesmo que tenha muitos autores. Por exemplo ``Johnson e Smith [Johnson and Smith1995] relata que...''
Evite a frase `` os autores afirmam que X''. O uso de ``afirmam'' lan\c{c}a d\'{u}vida em ``X'' porque referencia os pensamentos do autor ao inv\'{e}s dos fatos. Se você concorda ``X'' est\'{a} correto, simplesmente escreva ``X'' seguido da referência. Se absolutamente deve referenciar um artigo ao inv\'{e}s do resultado, diga ``o artigo afirma que'' ou ``Johnson e Smith [Johnson and Smith1995] apresentam evidências que...''

\section{Conceito Vs. Exemplo}

Um leitor pode ficar confuso quando um conceito e um exemplo deste est\~{a}o emba\c{c}ados. 
Exemplos comuns incluem: um algoritmo e um programa particular que o implementa, uma linguagem de programa\c{c}\~{a}o e um compilador, uma abstra\c{c}\~{a}o geral e sua implementa\c{c}\~{a}o 
particular em um sistema de computador, uma estrutura de dados e uma inst\^{a}ncia particular em mem\'{o}ria.

\section{Terminologia para Conceitos e Abstra\c{c}\~{o}es}

Quando definir a terminologia para um conceito, tenha cuidado para decidir precisamente como a id\'{e}ia se traduz para uma implementa\c{c}\~{a}o. Considere a seguinte discuss\~{a}o:
\textit{Sistemas VM incluem um conceito conhecido como endere\c{c}o de espa\c{c}o. O sistema cria dinamicamente um endere\c{c}o de espa\c{c}o quando um programa precisa de um, e destr\'{o}i um endere\c{c}o de espa\c{c}o 
quando o programa que criou o espa\c{c}o terminar de us\'{a}-lo. Um sistema VM usa um pequeno, finito n\'{u}mero para identificar cada endere\c{c}o de espa\c{c}o. 
Conceitualmente, entende-se que cada endere\c{c}o de espa\c{c}o deveria ter um novo identificador. Entretanto, se um sistema VM executa por um tempo que esgote todos os poss\'{i}veis identificadores de endere\c{c}os de espa\c{c}o, 
ele deve reusar um n\'{u}mero}
O ponto importante \'{e} que a discuss\~{a}o s\'{o} faz sentido porque define ``endere\c{c}o de espa\c{c}o'' independente de ``identificador de endere\c{c}o de espa\c{c}o''. 
Se espera-se discutir as diferen\c{c}as entre um conceito e sua implementa\c{c}\~{a}o, as defini\c{c}\~{o}es devem permitir tal distin\c{c}\~{a}o.

\section{Conhecimento Vs. Dados}

O fato que resulta de um experimento \'{e} chamado ``dado''. 
O termo ``conhecimento'' implica que o fato tenha sido analisado, condensado ou combinado com fatos de outros experimentos para produzir informa\c{c}\~{a}o \'{u}til.

\section{Causa e Efeito}

Uma monografia deve separar cuidadosamente causa-efeito de simples correla\c{c}\~{o}es estat\'{i}sticas. Por exemplo, mesmo se todos os programas de computador escritos no laborat\'{o}rio do Professor X requerem mais mem\'{o}ria que os programas escritos no laborat\'{o}rio do Professor Y, isso pode n\~{a}o ter nada a ver com os professores ou laborat\'{o}rios ou programadores (e.x., talvez as pessoas que trabalham no laborat\'{o}rio do Professor X estejam trabalhando em aplica\c{c}\~{o}es que requerem mais mem\'{o}ria do que no laborat\'{o}rio do Professor Y).

\section{Descreva Somente Conclus\~{o}es Comprovadas}   		

Deve-se ter cuidado para apenas escrever conclus\~{o}es que as evidências suportam. Por exemplo, se programas executam muito mais lento no computador A do que no computador B, 
n\~{a}o pode-se concluir que o processador de A \'{e} mais lento que o de B a menos que se tenha anotado todas as diferen\c{c}as entre os sistemas operacionais dos computadores, dispositivos de entrada e sa\'{i}da, 
tamanho de mem\'{o}ria, mem\'{o}ria cache, ou largura de banda do barramento interno. 
De fato, deve-se ainda abster-se de julgamentos a menos que se tenha os resultados de um experimento controlado (e.x., executando uma lista de v\'{a}rios programas muitas vezes, cada um quando o computador 
estiver ocioso). Mesmo se a causa de algum fenômeno parece \'{o}bvia, n\~{a}o pode-se dar uma conclus\~{a}o sem s\'{o}lida evidência embasada.

\section{Com\'{e}rcio e Ciência}

Em uma monografia, nunca se escreve conclus\~{o}es sobre viabilidade econômica ou sucesso comercial de uma id\'{e}ia/m\'{e}todo, nem faz-se especula\c{c}\~{o}es sobre a hist\'{o}ria do desenvolvimento ou origens de
uma id\'{e}ia. Um engenheiro deve permanecer objetivo sobre os m\'{e}ritos de uma id\'{e}ia, independente de sua popularidade comercial. Em particular, um engenheiro nunca assume que o sucesso comercial \'{e} uma medida v\'{a}lida 
de m\'{e}rito (muitos produtos populares n\~{a}o s\~{a}o nem bem projetados nem bem constru\'{i}dos). Assim, afirma\c{c}\~{o}es tais como ``mais de quatrocentos vendedores fazem produtos usando a t\'{e}cnica Y'' s\~{a}o irrelevantes em uma monografia.

\section{Pol\'{i}tica e Ciência}

Um engenheiro evita toda influência pol\'{i}tica quando est\'{a} avaliando Id\'{e}ias. Obviamente, n\~{a}o deveria importar se grupos governamentais, grupos pol\'{i}ticos, grupos religiosos ou outras organiza\c{c}\~{o}es aprovam uma id\'{e}ia.
Mais importante e frequentemente despercebida, n\~{a}o importa se uma id\'{e}ia uma id\'{e}ia foi originada por um engenheiro que j\'{a} tenha ganho um premio Nobel ou um aluno no primeiro ano de gradua\c{c}\~{a}o. Deve-se avaliar a id\'{e}ia independente da fonte.

\section{Organiza\c{c}\~{o}es Canônicas}
 	
Em geral toda monografia deve definir o problema que motivou a pesquisa, contar por que este problema \'{e} importante, contar o que outros fizeram, descrever as novas contribui\c{c}\~{o}es, documentar os experimentos que validam a contribui\c{c}\~{a}o e fazer conclus\~{o}es.
N\~{a}o existe organiza\c{c}\~{a}o canônica para uma monografia; cada uma \'{e} \'{u}nica. Entretanto, novatos que escrevem uma monografia em uma \'{a}rea experimental da Engenharia da Computa\c{c}\~{a}o podem achar os seguintes exemplos um bom ponto de in\'{i}cio: 

\begin{itemize}
	
	\item \textbf{Cap\'{i}tulo 1: Introdu\c{c}\~{a}o}

	Uma vis\~{a}o do problema; por que isso \'{e} importante; um resumo de um trabalho j\'{a} existente e uma afirma\c{c}\~{a}o de suas hip\'{o}teses ou quest\~{o}es espec\'{i}ficas a serem exploradas. Fa\c{c}a com que seja leg\'{i}vel pra qualquer um.

	\item \textbf{Cap\'{i}tulo 2: Defini\c{c}\~{o}es}

	Somente termos novos. Fa\c{c}a as defini\c{c}\~{o}es precisas, concisas e n\~{a}o-amb\'{i}guas.

	\item \textbf{Cap\'{i}tulo 3: Modelo Conceitual}

	Descreva o conceito central que influencia o seu trabalho. Fa\c{c}a disso um ``tema'' que amarram todos os seus argumentos. Isso deveria fornecer uma resposta para a quest\~{a}o apresentada na introdu\c{c}\~{a}o em um n\'{i}vel conceitual. Se necess\'{a}rio, adicione outro cap\'{i}tulo para dar um racioc\'{i}nio adicional sobre o problema ou sua solu\c{c}\~{a}o.

	\item \textbf{Cap\'{i}tulo 4: Medidas Experimentais}

	Descreva o resultado experimental que forne\c{c}a evidências para embasar sua tese. Normalmente experimentos enfatizam prova-ou-conceito (demonstrando a viabilidade de um m\'{e}todo/t\'{e}cnica) ou eficiência (demonstrando que um m\'{e}todo/t\'{e}cnica proporciona uma performance melhor do que as que j\'{a} existem)
	
	\item \textbf{Cap\'{i}tulo 5: Resultados e Consequências}

	Descreva varia\c{c}\~{o}es, extens\~{o}es ou outras aplica\c{c}\~{o}es da id\'{e}ia central.

	\item \textbf{Cap\'{i}tulo 6: Conclus\~{o}es}

	Resumo do que foi aprendido e como isso pode ser aplicado. Mencione as possibilidades para pesquisas futuras.

	\item \textbf{Resumo/Abstract}

	Um pequeno (poucos par\'{a}grafos) resumo da monografia. Descreva o problema e a abordagem da pesquisa. Enfatize as contribui\c{c}\~{o}es originais.

\end{itemize}

\section{Ordem Sugerida para Escrever}

A maneira mais f\'{a}cil de construir uma monografia \'{e} de dentro para fora. Comece escrevendo os cap\'{i}tulos que descrevem sua pesquisa (3, 4 e 5 nas linhas acima). Colete termos como eles surgem no texto e dê uma defini\c{c}\~{a}o para cada um.
Defina cada termo t\'{e}cnico,  mesmo que você o use de maneira convencional.

Organize as defini\c{c}\~{o}es em um cap\'{i}tulo separado. Fa\c{c}a as defini\c{c}\~{o}es precisas e formais. Reveja depois os cap\'{i}tulos para verificar que cada uso de termo t\'{e}cnico adere à uma defini\c{c}\~{a}o. Depois de ler os cap\'{i}tulos do meio para verificar terminologia, escreva a conclus\~{a}o. escreva a Introdu\c{c}\~{a}o logo depois da Conclus\~{a}o. Finalmente, complete o resumo/abstract.

\section{A Chave do Sucesso}

Ali\'{a}s, existe uma chave para o sucesso: pr\'{a}tica. ningu\'{e}m nunca aprendeu a escrever lendo composi\c{c}\~{o}es como esta. Ao inv\'{e}s disso, você precisa praticar, praticar, praticar. Todo dia.

\section{Pensamentos de Despedida}

Nos despedimos de você com as seguintes Id\'{e}ias para meditar. Se n\~{a}o significarem nada para você agora, visite-as novamente depois de ter escrito sua monografia.

	

	\indent\indent Depois de grande dor, chega um pensamento formal.

		\indent \indent \indent --Emily Dickinson

		
	\indent\indent Um homem pode escrever a qualquer hora, se ele se mantiver persistente para tal.

		\indent \indent \indent --Samuel Johnson

		
	\indent\indent Permane\c{c}a perfeito at\'{e} o final da estrada.

		\indent \indent \indent --Harry Lauder

	 
	\indent\indent Uma t\'{i}pica tese de Ph.D. \'{e} nada mais que transferir ossos de um cemit\'{e}rio para outro.

		\indent \indent \indent --Frank J. Dobie

	
	
	
	
	
	
	
	








	
	
	
	
	
	
	
	
	
	
	
	
	
	
