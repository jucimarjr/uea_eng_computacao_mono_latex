\chapter{Manual do Trabalho de Conclus\~{a}o de Curso - TCC - I e II}
\label{cap:manual TCC}


\section{Trabalho de Conclus\~{a}o de Curso (TCC)}

O TCC \'{e} o \'{u}ltimo e mais importante trabalho de disciplina a ser desenvolvido pelo aluno, individualmente, 
no Curso de Gradua\c{c}\~{a}o de Engenharia da Computa\c{c}\~{a}o. O aluno, orientado por um professor, ter\'{a} oportunidade de demonstrar e por em pr\'{a}tica 
os conhecimentos adquiridos durante o curso, al\'{e}m de aperfei\c{c}oar e comprovar o aprendizado te\'{o}rico e metodologias que lhe foram ensinadas.

O TCC consiste em duas etapas a serem realizadas em semestres distintos:

	\begin{enumerate}
		\item \textbf{TCC}-I: Elabora\c{c}\~{a}o e defesa da monografia.
		O aluno deve documentar, na forma de monografia, toda a fundamenta\c{c}\~{a}o te\'{o}rica e modelagem (se aplic\'{a}vel), de acordo com o tema proposto.
		
		\item \textbf{TCC}-II: Ap\'{o}s aprova\c{c}\~{a}o do TCC-I o aluno dever\'{a} concluir a monografia ao implementar a modelagem proposta ou outro instrumento que possibilite a avalia\c{c}\~{a}o do trabalho face aos objetivos definidos. A defesa inclui a documenta\c{c}\~{a}o finalizada e demonstra\c{c}\~{a}o do que foi desenvolvido (implementa\c{c}\~{a}o).
	\end{enumerate}
	
O TCC, do curso de Engenharia da Computa\c{c}\~{a}o, corresponde a um total de quatro (04) cr\'{e}ditos, a serem efetivados mediante matr\'{i}cula, sendo que dois (02) cr\'{e}ditos ser\~{a}o cumpridos na disciplina TCC-I e dois (02) cr\'{e}ditos na disciplina  TCC-II. 


\subsection{Do Professor da Disciplina}

O TCC funciona como uma disciplina e, portanto, possui um professor respons\'{a}vel, o qual encaminha para avalia\c{c}\~{a}o os trabalhos finais (TCC-I e TCC-II) e controla as notas finais. 

Compete ao professor da disciplina:

	\begin{itemize}
		\item Estabelecer reuni\~{a}o inicial com os alunos e professores para expor as normas do TCC e dar-lhes ci\^{e}ncia desse documento

		\item Divulgar e fazer cumprir as normas referentes ao TCC
		
		\item Divulgar o calend\'{a}rio referente \`{a}s atividades do TCC

		\item Penalizar o aluno pelo n\~{a}o cumprimento dos prazos determinados no calend\'{a}rio de atividades do TCC; (0,2 pts, na m\'{e}dia final, por dia de atraso na entrega 		da proposta e monografia)
		
		\item Divulgar aos alunos a rela\c{c}\~{a}o de professores e suas respectivas linhas de pesquisa para exercerem a atividade de orienta\c{c}\~{a}o
		
		\item Coordenar a forma\c{c}\~{a}o das duplas orientador/orientando

		\item Coordenar o processo de substitui\c{c}\~{a}o orientador/orientando
		
		\item Coordenar o processo de constitui\c{c}\~{a}o das bancas e definir o cronograma de apresenta\c{c}\~{a}o dos trabalhos
		
		\item Encaminhar aos membros da banca o respectivo trabalho para avalia\c{c}\~{a}o
		
		\item Convidar, quando poss\'{i}vel, um avaliador externo para integrar a banca
		
		\item Providenciar os documentos necess\'{a}rios ao processo de avalia\c{c}\~{a}o

		\item Providenciar as declara\c{c}\~{o}es aos professores participantes da banca, bem como do orientador
		
		\item Encaminhar uma c\'{o}pia definitiva do trabalho (TCC-II) \'{a} Biblioteca
		
	\end{itemize}


\subsection{Professor Orientador}

O professor orientador tem a fun\c{c}\~{a}o de ajudar o aluno no direcionamento do seu trabalho, sem, entretanto, desenvolver partes desse 
trabalho para o aluno. O orientador, apenas sugere caminhos que o aluno dever\'{a} seguir, acompanha seu trabalho, motivando e corrigindo eventuais erros.

Antes de apresentar o TCC-I ou TCC-II, o aluno deve submet\^{e}-lo previamente, \textbf{e obrigatoriamente}, \`{a} aprecia\c{c}\~{a}o de seu orientador. 
Dado o aval do mesmo, a proposta poder\'{a} ser encaminhada e apresentada para avalia\c{c}\~{a}o.

Compete ao Professor Orientador:

	\begin{itemize}
		\item Informar ao professor da disciplina a linha de pesquisa que ir\'{a} atuar
	
		\item Orientar a elabora\c{c}\~{a}o do Trabalho de Conclus\~{a}o
	
		\item Auxiliar o aluno na resolu\c{c}\~{a}o de problemas conceituais, t\'{e}cnicos e de relacionamento decorrentes da atividade
	
		\item Estabelecer o plano e cronograma de trabalho em conjunto com o orientando
	
		\item Informar o orientando sobre as normas, procedimentos e crit\'{e}rios de avalia\c{c}\~{a}o respectivos
	
		\item Informar ao aluno, caso haja atraso no cronograma de trabalho ou o n\~{a}o cumprimento das orienta\c{c}\~{o}es, se o trabalho tem condi\c{c}\~{o}es ou n\~{a}o de ser 
		encaminhado para avalia\c{c}\~{a}o
	
		\item liberar o trabalho para que haja a apresenta\c{c}\~{a}o do aluno bem como informar ao professor da disciplina quanto \`{a} apresenta\c{c}\~{a}o do aluno (Anexo C)
	
		\item Rubricar as 3 (tr\^{e}s) vias encaminhadas para avalia\c{c}\~{a}o (TCC-I ou TCC-II) quando estiver ciente e de acordo, 
		conforme suas orienta\c{c}\~{o}es, do material entregue
	
		\item Presidir a banca examinadora do trabalho por ele orientado
	
		\item Comunicar ao professor da disciplina situa\c{c}\~{o}es que exijam provid\^{e}ncias, assim que ocorrerem.
	\end{itemize}


\subsection{Co-orientador}

	\begin{itemize}	
		\item Ser\'{a} solicitado formalmente pelo orientador a Professor da disciplina e ser\'{a} designado por esse para atender quest\~{a}o espec\'{i}fica do trabalho
		
		\item Trabalhar\'{a} em conjunto ao orientador e desempenhar\'{a} papel solicitado pelo mesmo.
	\end{itemize}


\subsection{Orientando}

Compete ao Orientando:

	\begin{itemize}
		\item Comparecer \`{a}s reuni\~{o}es marcadas pelo professor da disciplina sobre o Trabalho de Conclus\~{a}o
		
		\item Escolher a tem\'{a}tica a ser trabalhada em conson\^{a}ncia com as linhas de pesquisa do curso
		
		\item Contatar professor para definir orientador e informar a Coordena\c{c}\~{a}o do Projeto (entrega do Anexo-B)
		
		\item Cumprir as datas limites determinadas no calend\'{a}rio de atividades do TCC. O n\~{a}o cumprimento dos prazos ser\'{a} penalizado com perda de pontua\c{c}\~{a}o; 
		(0,2 pts, na m\'{e}dia final, por dia de atraso na entrega da monografia)
		
		\item Comparecer \`{a}s orienta\c{c}\~{o}es sobre o trabalho; o n\~{a}o comparecimento de tr\^{e}s (03) orienta\c{c}\~{o}es seguidas implica em reprova\c{c}\~{a}o por falta
		
		\item Seguir as orienta\c{c}\~{o}es do professor designado \`{a} orienta\c{c}\~{a}o
		
		\item Cumprir o plano e o cronograma de trabalho elaborado em conjunto com o professor-orientador
		
		\item Comunicar ao professor da disciplina toda e qualquer situa\c{c}\~{a}o que possa comprometer, de alguma forma, o processo de elabora\c{c}\~{a}o, bem como, a conclus\~{a}o do 			trabalho o quanto antes, para que a coordena\c{c}\~{a}o possa analisar o ocorrido e tomar as provid\^{e}ncias cab\'{i}veis
		
		\item Comparecer perante a banca na data, hora e local estabelecido para a realiza\c{c}\~{a}o da sess\~{a}o de avalia\c{c}\~{a}o do TCC
	\end{itemize}


\subsection{Os Acompanhamentos de Orienta\c{c}\~{a}o}

As reuni\~{o}es de orienta\c{c}\~{a}o dever\~{a}o ser documentadas conforme modelo presente no Anexo A e ser\~{a}o entregues ao professor da disciplina no dia da entrega da carta (ANEXO A) solicitando defesa de TCC.
Tanto professor orientador como orientando dever\~{a}o ter uma c\'{o}pia dos acompanhamentos de orienta\c{c}\~{a}o.
	

\subsection{A Banca Examinadora}

A banca examinadora do TCC-I e TCC-II, dever\'{a} ser composta por, no m\'{i}nimo, 3 professores. A banca ser\'{a} constitu\'{i}da pelo 
professor orientador e por dois outros professores. Se houver co-orienta\c{c}\~{a}o, o professor co-orientador pode compor a banca, contudo sua avalia\c{c}\~{a}o n\~{a}o computar\'{a} nota para o alunosua avalia\c{c}\~{a}o n\~{a}o computar\'{a} nota para o aluno 

Os membros da banca examinadora poder\~{a}o sugerir altera\c{c}\~{o}es no trabalho (parte escrita e/ou implementa\c{c}\~{a}o). Para o TCC-I as altera\c{c}\~{o}es dever\~{a}o ser feitas, com o acompanhamento do orientador, para que sejam inclu\'{i}das no trabalho e avaliadas no TCC-II. Para o TCC-II, estas dever\~{a}o ser feitas at\'{e} duas semanas depois da apresenta\c{c}\~{a}o (ver data limite), supervisionadas pelo professor-orientador, para constar no(s) volume(s) final(is) do TCC, que ficar\'{a} \`{a} disposi\c{c}\~{a}o na biblioteca

O volume final para arquivamento (TCC-II) s\'{o} ser\'{a} aceito pela coordena\c{c}\~{a}o de TCC se estiver validado pelo professor orientador, indicando sua concord\^{a}ncia com o conte\'{u}do do mesmo, e a assinatura do aluno


\subsection{Semin\'{a}rios}
Conforme calend\'{a}rio os semin\'{a}rios destinados aos alunos matriculados em TCC, abordam temas que auxiliar\~{a}o na elabora\c{c}\~{a}o do documento escrito e na defesa.

Ser\~{a}o e semin\'{a}rios:
	\begin{itemize}
		\item Semin\'{a}rio I - Estrutura do Trabalho de Conclus\~{a}o de Curso

		\item Semin\'{a}rio II - Normas ABNT

		\item Semin\'{a}rio III - Apresenta\c{c}\~{a}o do TCC (defesa e material)
	\end{itemize}

A participa\c{c}\~{a}o do aluno nos semin\'{a}rios \'{e} um dos crit\'{e}rios que consta na ata de avalia\c{c}\~{a}o. O n\~{a}o comparecimento acarretar\'{a} perda de 0,2 pt por semin\'{a}rio.


\subsection{As Datas Limite}

As datas limites ser\~{a}o estabelecidas e divulgadas de acordo com o calend\'{a}rio acad\^{e}mico de cada per\'{i}odo acad\^{e}mico.


\subsection{Nota Final}

Para aprova\c{c}\~{a}o do aluno no TCC, o mesmo dever\'{a}:

	\begin{itemize}
		\item Atender \`{a} exig\^{e}ncia da frequ\^{e}ncia m\'{i}nima de 75$\%$ (setenta e cinco) \`{a}s orienta\c{c}\~{o}es. 
		A frequ\^{e}ncia do aluno ser\'{a} validada a partir do formul\'{a}rio de acompanhamento de reuni\~{a}o de orienta\c{c}\~{a}o os quais devem ser preenchidos a cada acompanhamento, 			pelo orientador e pelo aluno

		\item Obter, no m\'{i}nimo, grau 6,0 (seis). Este grau ser\'{a} composto pela m\'{e}dia aritm\'{e}tica das avalia\c{c}\~{o}es dos membros da banca examinadora. Cada membro da banca 			receber\'{a} uma planilha com itens a avaliar (por notas). Ao t\'{e}rmino da defesa ser\'{a} preenchida uma ata final de avalia\c{c}\~{a}o de TCC constando a m\'{e}dia final do aluno.
		Caso o aluno n\~{a}o alcance grau m\'{i}nimo seis (6,0) dever\'{a} matricular-se novamente na disciplina para desenvolver novamente o trabalho ou concluir o 			desenvolvimento do mesmo
	\end{itemize}


\section{Trabalho de Conclus\~{a}o de Curso I (TCC-I)}

O aluno, em parceria com um professor orientador, deve delimitar um tema, a ser abordado, dentro das linhas de pesquisa do curso.

Deve ent\~{a}o dar in\'{i}cio \`{a} documenta\c{c}\~{a}o de seu trabalho elaborando uma monografia com os cap\'{i}tulos contendo a fundamenta\c{c}\~{a}o te\'{o}rica e modelagem da implementa\c{c}\~{a}o do trabalho proposto. Ao final do semestre defend\^{e}-lo \`{a} uma banca examinadora.

\subsection{Estrutura do TCC-I}

Na monografia, o aluno dever\'{a} documentar seu trabalho para ser arquivado e, no futuro, referenciado por outras pessoas, lembrando sempre que Trabalho de Conclus\~{a}o de Curso deve ser escrito tendo em vista uma metodologia cient\'{i}fica. 

A monografia deve seguir a seguinte estrutura: 
	
	\indent \textbf{Capa} \\
	\indent \textbf{Folha de Rosto} \\
	\indent \textbf{Ficha para Cataloga\c{c}\~{a}o} (deve ser impressa no verso da folha de rosto) \\
	\indent \textbf{Ep\'{i}grafe} (opcional) \\
	\indent \textbf{Dedicat\'{o}ria} (opcional) \\
	\indent \textbf{Agradecimentos} (opcional) \\
	\indent \textbf{Resumo} \\
	\indent \textbf{Abstract} (resumo em ingl\^{e}s) \\
	\indent \textbf{Sum\'{a}rio} \\
	\indent \textbf{Lista de Figuras} (opcional) \\
	\indent \textbf{Lista de Tabelas} (opcional) \\
	\indent \textbf{Lista de Abreviaturas e siglas} \\
	\indent \textbf{Introdu\c{c}\~{a}o} \\
	\indent \textbf{Desenvolvimento} \\
	\indent \textbf{Conclus\~{a}o} \\
	\indent \textbf{Refer\^{e}ncias Bibliogr\'{a}ficas} \\
	\indent \textbf{Obras Consultadas} \\
	\indent \textbf{Anexos e/ou Ap\^{e}ndices} (opcional) \\\\

\textbf{Introdu\c{c}\~{a}o} - \'{e} o primeiro cap\'{i}tulo da monografia. Apresenta o contexto do trabalho proposto com a defini\c{c}\~{a}o do problema, os objetivos (geral e espec\'{i}ficos), os motivos que levaram \`{a} decis\~{a}o de se abordar o tema e a organiza\c{c}\~{a}o do trabalho.

\textbf{Desenvolvimento} - corresponde aos demais cap\'{i}tulos da monografia, que descrevem sobre o tema proposto, revis\~{a}o da literatura, metodologia aplicada, ferramentas e modelagem (se aplic\'{a}vel) do trabalho a ser implementado.

\textbf{Conclus\~{a}o} - como o TCC-I \'{e} o in\'{i}cio da monografia n\~{a}o ser\'{a} poss\'{i}vel uma conclus\~{a}o, portanto devem ser apresentadas as dificuldades encontradas, at\'{e} o momento no trabalho, e resultados esperados do trabalho proposto.

A formata\c{c}\~{a}o (margens, espa\c{c}amentos, cita\c{c}\~{o}es, pagina\c{c}\~{a}o, etc.) de todo o documento, deve estar voltada para um trabalho cient\'{i}fico, portanto, 
os alunos devem seguir o modelo de monografia adotado pelo curso e dispon\'{i}vel no site do mesmo. 
Solicitamos ainda aos alunos que utilizem as obras abaixo:

	\begin{itemize}	
		\item  FURAST\'{E}, Pedro Augusto. Normas T\'{e}cnicas para o Trabalho Cient\'{i}fico (Nova ABNT). 14 edi\c{c}\~{a}o. Porto Alegre, 2006.

		\item SILVA, Edna L\'{u}cia da. Metodologia da Pesquisa e Elabora\c{c}\~{a}o de Disserta\c{c}\~{a}o - 3ed. rev. e atual. – Florian\'{o}polis: Laborat\'{o}rio de Ensino a Dist\^{a}ncia da 			UFSC, 2001.

		\item BRASIL, ABNT - Associa\c{c}\~{a}o Brasileira de Normas T\'{e}cnicas. NBR 14724.

		\item BRASIL, ABNT - Associa\c{c}\~{a}o Brasileira de Normas T\'{e}cnicas. NBR 10520.

		\item BRASIL, ABNT - Associa\c{c}\~{a}o Brasileira de Normas T\'{e}cnicas. NBR 6023.
	\end{itemize}
	

\subsection{Avali\c{c}\~{a}o do TCC-I}

O TCC-I dever\'{a} ser apresentado perante uma banca examinadora a ser definida pelo professor da disciplina, para a qual o aluno apresentar\'{a} seu trabalho, desde a justificativa do problema que o levou a desenvolv\^{e}-lo at\'{e} as discuss\~{o}es do material levantado.

O aluno ter\'{a} 25 (vinte e cinco) minutos para defesa de sua proposta, onde utilizar\'{a} os recursos audiovisuais que achar necess\'{a}rio e ser\~{a}o utilizados mais 10 (dez) minutos para responder aos questionamentos de cada membro da banca avaliadora. A banca ser\'{a} constitu\'{i}da pelo professor orientador (presidente) e por dois outros professores, podendo ser um convidado externo. Avaliado o trabalho escrito e ``ouvidas'' as sugest\~{o}es da banca, o aluno dever\'{a} fazer as modifica\c{c}\~{o}es necess\'{a}rias.

No TCC-I caso o aluno n\~{a}o alcance grau m\'{i}nimo 6,0 (seis) dever\'{a} matricular-se novamente na disciplina pra desenvolver novamente o trabalho (ou concluir o desenvolvimento do mesmo)  em TCC-I.
\'{e} necess\'{a}rio que o aluno seja aprovado em TCC-I para a conclus\~{a}o do trabalho em TCC-II. 


\section{Trabalho de Conclus\~{a}o de Curso II (TCC-II)}

O aluno dever\'{a} por em pr\'{a}tica a modelagem apresentada em TCC-I al\'{e}m de concluir a monografia (implementa\c{c}\~{a}o, resultados obtidos, etc.). Haver\'{a} nova defesa da documenta\c{c}\~{a}o e demonstra\c{c}\~{a}o do que foi desenvolvido (implementa\c{c}\~{a}o) de acordo com esta documenta\c{c}\~{a}o.


\subsection{Estrutura do TCC-II}

Apresenta a mesma estrutura do TCC-I. Entretanto, no TCC-II, o aluno ir\'{a} complementar a monografia de acordo com as solicita\c{c}\~{o}es feitas pela banca examinadora (na defesa do TCC-I) e com t\'{o}picos relacionados \`{a} sua implementa\c{c}\~{a}o. 

A monografia deve seguir a seguinte estrutura: 

	\indent \textbf{Capa} \\
	\indent \textbf{Folha de Rosto} \\
	\indent \textbf{Ficha para Cataloga\c{c}\~{a}o} (deve ser impressa no verso da folha de rosto) \\
	\indent \textbf{Ep\'{i}grafe} (opcional) \\
	\indent \textbf{Dedicat\'{o}ria} (opcional) \\
	\indent \textbf{Agradecimentos} (opcional) \\
	\indent \textbf{Resumo} \\
	\indent \textbf{Abstract} (resumo em ingl\^{e}s) \\
	\indent \textbf{Sum\'{a}rio} \\
	\indent \textbf{Lista de Figuras} (opcional) \\
	\indent \textbf{Lista de Tabelas} (opcional) \\
	\indent \textbf{Lista de Abreviaturas e siglas} \\
	\indent \textbf{Introdu\c{c}\~{a}o} \\
	\indent \textbf{Desenvolvimento} \\
	\indent \textbf{Conclus\~{a}o} \\
	\indent \textbf{Refer\^{e}ncias Bibliogr\'{a}ficas} \\
	\indent \textbf{Obras Consultadas} \\
	\indent \textbf{Anexos e/ou Ap\^{e}ndices} (opcional) \\\\


\textbf{Introdu\c{c}\~{a}o} - \'{e} o primeiro cap\'{i}tulo da monografia. Apresenta o contexto do trabalho proposto com a defini\c{c}\~{a}o do problema, os objetivos (geral e espec\'{i}ficos), os motivos que levaram \`{a} decis\~{a}o de se abordar o tema e a organiza\c{c}\~{a}o do trabalho.

\textbf{Desenvolvimento} - corresponde aos demais cap\'{i}tulos da monografia, que descrevem sobre o tema proposto, revis\~{a}o da literatura, metodologia aplicada, ferramentas e modelagem (se aplic\'{a}vel) do trabalho a ser implementado.

\textbf{Conclus\~{a}o} - se os objetivos foram atingidos, dificuldades encontradas e sugest\~{o}es para trabalhos futuros.

A formata\c{c}\~{a}o (margens, espa\c{c}amentos, cita\c{c}\~{o}es, pagina\c{c}\~{a}o, etc.) de todo o documento, deve estar voltada para um trabalho cient\'{i}fico, portanto, 
os alunos devem seguir o modelo de monografia adotado pelo curso e dispon\'{i}vel no site do mesmo. 
Solicitamos ainda aos alunos que utilizem as obras abaixo:

	\begin{itemize}	
		\item  FURAST\'{e}, Pedro Augusto. Normas T\'{e}cnicas para o Trabalho Cient\'{i}fico (Nova ABNT). 14 edi\c{c}\~{a}o. Porto Alegre, 2006.

		\item SILVA, Edna L\'{u}cia da. Metodologia da Pesquisa e Elabora\c{c}\~{a}o de Disserta\c{c}\~{a}o - 3ed. rev. e atual. - Florian\'{o}polis: Laborat\'{o}rio de Ensino a Dist\^{a}ncia da 			UFSC, 2001.

		\item BRASIL, ABNT - Associa\c{c}\~{a}o Brasileira de Normas T\'{e}cnicas. NBR 14724.

		\item BRASIL, ABNT - Associa\c{c}\~{a}o Brasileira de Normas T\'{e}cnicas. NBR 10520.

		\item BRASIL, ABNT - Associa\c{c}\~{a}o Brasileira de Normas T\'{e}cnicas. NBR 6023.
	\end{itemize}


\subsection{Avalia\c{c}\~{a}o do TCC-II}

O TCC-II dever\'{a} ser apresentado perante uma banca examinadora a ser definida pelo professor da disciplina, para a qual o aluno apresentar\'{a} seu trabalho, desde a justificativa do problema que o levou a desenvolv\^{e}-lo at\'{e} as discuss\~{o}es do material levantado e a conclus\~{a}o.

Esta apresenta\c{c}\~{a}o dever\'{a} ser, necessariamente, oral e descritiva, onde o aluno dever\'{a} tamb\'{e}m na parte oral resumir as principais fun\c{c}\~{o}es do sistema, o modo como ser\'{a} usado na organiza\c{c}\~{a}o ou ambiente e os seus benef\'{i}cios.

Para esta apresenta\c{c}\~{a}o oral, o aluno dever\'{a} preparar o que ir\'{a} falar e utilizar recursos did\'{a}ticos, considerando o tempo de 45 minutos; cada membro da banca avaliadora ter\'{a} 10 minutos para questionamentos.

N\~{a}o ser\~{a}o aceitas justificativas para a n\~{a}o demonstra\c{c}\~{a}o das implementa\c{c}\~{o}es, implicando assim em reprova\c{c}\~{a}o.

As apresenta\c{c}\~{o}es dos TCCs s\~{a}o abertas ao p\'{u}blico interessado. Sugere-se, fortemente, que os alunos de TCC-I assistam \`{a}s bancas de seus colegas de TCC-II, como experi\^{e}ncia.

Caso o aluno n\~{a}o alcance a nota m\'{i}nima de 6,0 pontos em TCC-II, dever\'{a} matricular-se, novamente na disciplina, no pr\'{o}ximo semestre.
	
A nota do aluno s\'{o} ser\'{a} lan\c{c}ada mediante entrega de 2 c\'{o}pias da vers\~{a}o revisada com visto do orientador, j\'{a} incluindo as modifica\c{c}\~{o}es sugeridas pela banca, no formato final (impresso em jato de tinta ou laser), em capa dura, na cor preta, bem como um CD com a vers\~{a}o final do trabalho e os produtos resultantes da pesquisa, quando for o caso.

	
